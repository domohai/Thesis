\documentclass[a4paper]{report}
\usepackage{amsmath}

\usepackage{algpseudocode}
\usepackage[utf8]{inputenc}
\usepackage{vntex}
\usepackage[english]{babel}
\usepackage[table]{xcolor}
% \usepackage{enumitem}
\usepackage{amsfonts}
\usepackage{amssymb}
\usepackage{graphicx}
\usepackage[hang,flushmargin]{footmisc} % For hanging footnotes
\usepackage{setspace} % For fine spacing adjustments
\usepackage[font=normalsize]{caption}
\usepackage{booktabs}
\usepackage[unicode]{hyperref}
\usepackage[left=3cm,right=2cm,top=2.5cm,bottom=3cm]{geometry}
\usepackage{titlesec}
\usepackage{mdframed}
\usepackage{lipsum} % For example text (if needed)
\usepackage{colortbl}
\usepackage{booktabs}
\usepackage{array}
\usepackage{xcolor}
\usepackage{scrextend}
\usepackage{enumerate}
\usepackage{tikz}
\usepackage{float}
\usepackage{afterpage}
\usepackage{multirow}
\usepackage{multicol}
\usepackage{sectsty}
\usepackage{tocloft,calc}
\usepackage{listings}
\usepackage{makecell}
\usepackage{tocloft}
\usepackage{tcolorbox}
\usepackage{mathptmx}
\usepackage{longtable} % Include this in the preamble
\usepackage{inconsolata}  % Font monospace đẹp cho code
\usepackage{csquotes}
\usepackage{lmodern}
\usepackage{tabularx} % Add to preamble if not already there

\hypersetup{pdfborder={0 0 0}}

% Định nghĩa màu
\definecolor{orange}{RGB}{255,128,0}
\definecolor{gray}{RGB}{128,128,128}
\definecolor{purple}{RGB}{128,0,128}

\usetikzlibrary{calc}
\usepackage{algorithm}
\usepackage{perpage} %the perpage package
\MakePerPage{footnote} %the perpage package command
\PassOptionsToPackage{table}{xcolor}

\def\changemargin#1#2{\list{}{\rightmargin#2\leftmargin#1}\item[]}
\let\endchangemargin=\endlist

\changefontsizes{13pt}
% \bibliographystyle{unsrt}

\lstset{
  basicstyle=\ttfamily\scriptsize, % Match font size and style in the image
  breaklines=true, % Allow line breaks
  numbers=left, % Add line numbers to the left
  % numberstyle=\tiny, % Smaller font for line numbers
  frame=lines, % Add a frame around the code
  xleftmargin=20pt, % Adjust left margin for alignment
  tabsize=4, % Tab size for indentation
  showstringspaces=false, % Hide visible spaces in strings
  language=Prolog % Set the language to Prolog
}

%%% Code block style
\definecolor{codebackground}{rgb}{0.95,0.95,0.97}
\definecolor{codekeyword}{rgb}{0.13,0.29,0.53}
\definecolor{codecomment}{rgb}{0.25,0.5,0.35}
\definecolor{codestring}{rgb}{0.63,0.13,0.09}
\lstdefinestyle{toclstyle}{
  backgroundcolor=\color{codebackground},
  basicstyle=\ttfamily\scriptsize,
  breakatwhitespace=false,
  breaklines=true,
  commentstyle=\color{codecomment},
  keywordstyle=\color{codekeyword}\bfseries,
  stringstyle=\color{codestring},
  numbers=left,
  numbersep=8pt,
  numberstyle=\tiny\color{gray},
  frame=single,
  rulecolor=\color{gray!30},
  showspaces=false,
  showstringspaces=false,
  showtabs=false,
  tabsize=2,
  morekeywords={context, inv, always, next, sometime, until, before, previous, 
                alwaysPast, sometimePast, since, implies, and, or, not, self, 
                isCalled, becomesTrue},
}

% Java code listing style - Add to preamble
% Java code listing style - Update string color
\definecolor{javakeyword}{rgb}{0.0, 0.0, 0.75}     % Dark blue for keywords
\definecolor{javacomment}{rgb}{0.5, 0.5, 0.5}      % Gray for comments
\definecolor{javastring}{rgb}{0.0, 0.6, 0.0}       % Changed to green for strings
\definecolor{javaannotation}{rgb}{0.6, 0.0, 0.6}   % Purple for annotations
\definecolor{javadoc}{rgb}{0.25, 0.5, 0.35}        % Forest green for javadoc
\definecolor{javabackground}{rgb}{0.98, 0.98, 1.0} % Very light blue background
\lstdefinestyle{javastyle}{
  language=Java,
  backgroundcolor=\color{white},
  basicstyle=\ttfamily\footnotesize,
  keywordstyle=\color{javakeyword}\bfseries,
  commentstyle=\color{javacomment}\itshape,
  stringstyle=\color{javastring},
  numberstyle=\tiny\color{gray},
  breakatwhitespace=false,
  breaklines=true,
  keepspaces=true,
  numbers=left,
  numbersep=5pt,
  frame=none,
  rulecolor=\color{black},
  showspaces=false,
  showstringspaces=false,
  showtabs=false,
  tabsize=4,
  % Java-specific keywords
  morekeywords={String, TokenStream},
  % Handle annotations
  moredelim=[s][\color{javaannotation}]{@}{(},
  % Handle Javadoc comments
  morecomment=[s][\color{javadoc}]{/*}{*/},
  morecomment=[l][\color{javacomment}]{//},
}

\makeatletter
\def\l@figure{\@dottedtocline{1}{1em}{2.2em}}
\def\l@table{\@dottedtocline{1}{1em}{2.2em}}
\renewcommand*{\ALG@name}{Algorithm}
\makeatother

\sectionfont{\fontsize{15}{15}\selectfont}
\subsectionfont{\fontsize{13}{15}\selectfont}
\titleformat{\subsubsection}[runin]
{\normalfont\normalsize\bfseries}{\thesubsubsection.}{1em}{}
\titlespacing*{\subsubsection}{0pt}{*0.5}{*0.5}

% Include subsubsections in numbering and TOC
\setcounter{secnumdepth}{3}
\setcounter{tocdepth}{3}

\titleformat{\chapter}[display]
{\normalfont\huge\bfseries}{\chaptertitlename\ \thechapter}{0pt}{\LARGE}
\titlespacing*{\chapter}{0cm}{-\topskip}{20pt}[0pt]

\renewcommand{\baselinestretch}{1.3}
%\renewcommand{\cftsecpresnum}{Chương\space}
\renewcommand{\cftchappresnum}{Chapter }
\AtBeginDocument{\addtolength\cftchapnumwidth{\widthof{\bfseries Chapter}}}
\setlength{\parskip}{0.4em}
\setlength{\parindent}{0pt}


% Tùy chỉnh định dạng mục lục
\renewcommand{\cftchappresnum}{Chapter }
\renewcommand{\cftchapaftersnum}{: }

\setlength{\cftchapnumwidth}{2.5em}

%
\newcommand{\argmax}{\arg\!\max}
\usepackage[
  backend=biber,
  sorting=ynt
]{biblatex}
\addbibresource{references.bib}


\usepackage{graphicx}
\usepackage{array}
\graphicspath{ {figures/} }

\renewcommand{\listfigurename}{List of plots}
\renewcommand{\listtablename}{Tables}

%%%%%%%%%%%%%%%%%%%%%%%%%%%%%%%%%%%%%%%%%%%%%%%%%%%%%%%%%%%%%%%%%%%%%%%%%%%%%%%%%%%%%%%%%%%%%%%%%%%%%%%
\begin{document}
\newcommand\setfontsize[1]{\fontsize{#1}{1.3\dimexpr#1}\selectfont}

% Cover english
\pagenumbering{gobble}

\begin{center}
	\begin{tikzpicture}[overlay,remember picture]
		\draw[line width=3pt,color=black,fill=none]
		($(current page.north west)+(2.5cm,-2cm)$) rectangle
		($(current page.south east)-(1.5cm,-2.5cm)$);
		\draw[line width=1pt,color=black,fill=none]
		($(current page.north west)+(2.65cm,-2.15cm)$)
		rectangle ($(current page.south east)-(1.65cm,-2.65cm)$);
	\end{tikzpicture}
	\\[1mm]
	\setfontsize{12pt}\textbf{VIETNAM NATIONAL UNIVERSITY, HANOI
		\\UNIVERSITY OF ENGINEERING AND TECHNOLOGY}\\[2cm]
	\includegraphics[width=0.2\linewidth]{figures/uet.jpg}\\[0.5cm]
	\setfontsize{14pt}\textbf{}
	\\[1.5cm]
	\setfontsize{18pt}\textbf{PHÁT TRIỂN HỆ THỐNG TẠO LẬP VÀ QUẢN LÝ GIAO
		DỊCH ĐỒNG TIỀN ĐIỆN TỬ PHI TẬP TRUNG  \\TRÊN NỀN TẢNG ĐA CHUỖI KHỐI}
	\\[2cm]
	\setfontsize{14pt}\textbf{KHÓA LUẬN TỐT NGHIỆP
		\\
		Chuyên ngành: Khoa học máy tính }

	\vfill
	\setfontsize{12pt}\textbf{HÀ NỘI – 2025}
\end{center}

\newpage

% \begin{center}
\begin{tikzpicture}[overlay,remember picture]
	\draw[line width=3pt,color=black,fill=none]
	($(current page.north west)+(2.5cm,-2cm)$) rectangle
	($(current page.south east)-(1.5cm,-2.5cm)$);
	\draw[line width=1pt,color=black,fill=none]
	($(current page.north west)+(2.65cm,-2.15cm)$)
	rectangle ($(current page.south east)-(1.65cm,-2.65cm)$);
\end{tikzpicture}
\\[1mm]
{
\centering
\setfontsize{12pt}\textbf{VIETNAM NATIONAL UNIVERSITY, HANOI
	\\UNIVERSITY OF ENGINEERING AND TECHNOLOGY}\\[1.5cm]
\setfontsize{14pt}\textbf{Trần Chiến Thắng}
\\[1.5cm]
\setfontsize{18pt}\textbf{PHÁT TRIỂN HỆ THỐNG TẠO LẬP VÀ QUẢN LÝ GIAO
	DỊCH ĐỒNG TIỀN ĐIỆN TỬ PHI TẬP TRUNG  \\TRÊN NỀN TẢNG ĐA CHUỖI KHỐI}
\\[3cm]
\setfontsize{14pt}\textbf{KHÓA LUẬN TỐT NGHIỆP \\
	Chuyên ngành: Khoa học máy tính}\\[3cm]
}

\hspace*{8mm}\setfontsize{14pt}\textbf{Giảng viên hướng dẫn: PSG.TS.
	Đặng Đức Hạnh}\\[0.8cm]

\begin{center}
	\vfill
	\setfontsize{12pt}\textbf{HÀ NỘI – 2025}
\end{center}
% \end{center}

% \begin{center}

% 	\begin{tikzpicture}[overlay,remember picture]
% 		\draw[line width=3pt,color=black,fill=none]
% 			($(current page.north west)+(2.5cm,-2cm)$) rectangle ($(current page.south east)-(1.5cm,-2.5cm)$);
% 		\draw[line width=1pt,color=black,fill=none]
% 			($(current page.north west)+(2.65cm,-2.15cm)$) rectangle ($(current page.south east)-(1.65cm,-2.65cm)$);
% 	\end{tikzpicture}
% 	\\[1mm]
% 	\textbf{VIETNAM NATIONAL UNIVERSITY, HANOI \\UNIVERSITY OF ENGINEERING AND TECHNOLOGY}
% 	\\[1cm]
% 	\\[0.3cm]
% 	\textbf{Dang Tran Hoang Ha}
% 	\\[2cm]

% 	\large{\textbf{A SAT APPROACH FOR SOLVING \\SOCIAL GOLFER PROBLEM - SGP}}
% 	\\[2.6cm]
% 	\normalsize{\textbf{BACHELOR’S THESIS
% 		\\[2mm]
% 	Major: Information Technology}}
% \end{center}
% \vspace{16mm}
% \hspace*{12mm}\textbf{ Supervisor: PhD. To Van Khanh}\\[0.8cm]

% \vfill
% \begin{center}
% 	\textbf{HA NOI – 2024}
% 	\vspace{10mm}
% \end{center}\newpage\cleardoublepage

% Abstract Vietnamese
% \begin{center}
  \textbf{\large{TÓM TẮT}	}
\end{center}

\addcontentsline{toc}{chapter}{TÓM TẮT}

\begin{small}

  \textbf{Tóm tắt:} Khóa luận tập trung nghiên cứu và phát triển ứng dụng
  LaunchCrypt,
  một nền tảng đổi mới trong lĩnh vực tài chính phi tập trung (DeFi), nhằm đơn
  giản hóa
  quá trình tạo và giao dịch token cho người dùng không có kiến thức chuyên sâu
  về
  công nghệ. Để xây dựng nền tảng này, khóa luận đề xuất sử dụng stack công
  nghệ hiện
  đại bao gồm ReactJS cho phần frontend, NestJS cho backend, và các ngôn ngữ
  lập
  trình hợp đồng thông minh như Move, Solidity, và Rust để hỗ trợ đa chuỗi. Bài toán
  chính mà
  khóa luận giải quyết là tạo ra một hệ sinh thái toàn diện cho việc tạo, giao
  dịch, và
  phát triển token một cách an toàn và hiệu quả. Để có thể hiểu và áp dụng đúng
  các
  công nghệ cho bài toán này, khóa luận trước tiên đã nêu một số kiến thức nền
  tảng về
  công nghệ chuỗi khối, hợp đồng thông minh, các nguyên lý hoạt động của DeFi, cũng như các
  giao thức
  và chuẩn token trên các chuỗi khối khác nhau. Sau khi tìm hiểu tất cả các
  kiến thức
  nền tảng liên quan, khóa luận vận dụng phát triển ba mô đun chính của
  LaunchCrypt:
  (1) mô đun tạo token, cho phép người dùng dễ dàng tạo ra token của riêng họ
  trên các
  chuỗi được hỗ trợ, (2) mô đun giao dịch token, tự động tạo cặp giao dịch
  và cung
  cấp giao diện cho việc mua bán token, và (3) mô đun tốt nghiệp token, quản lý
  quá
  trình chuyển token lên các sàn giao dịch phi tập trung lớn hơn khi đạt đủ
  thanh khoản.
  Bên cạnh đó, khóa luận cũng phát triển các tính năng bổ sung như stake token,
  khả năng tương tác, kết nối giữa các đối tượng người dùng cùng với khả năng hỗ
  trợ đa chuỗi, bao gồm Aptos, Solana và Avalanche. Điều này không chỉ mở rộng
  phạm vi tiếp cận của dự án mà còn thúc đẩy tính linh hoạt và khả năng tương tác
  giữa các chuỗi khối khác nhau. Bằng
  cách cung cấp các công cụ như giao diện Tradingview cho giao dịch nâng cao,
  tính năng
  mua/bán tự động và thông tin chi tiết về token, LaunchCrypt không chỉ phục vụ
  người
  dùng mới mà còn đáp ứng nhu cầu của các nhà giao dịch có kinh nghiệm. Hơn
  nữa,
  việc tích hợp với các giải pháp để hỗ trợ người dùng Web 2.0 thể hiện tầm
  nhìn dài hạn
  của dự án trong việc thu hẹp khoảng cách giữa tài chính truyền thống và DeFi.
  Cuối
  cùng, khóa luận sẽ đề cập cách triển khai và thử nghiệm nền tảng LaunchCrypt
  trên
  môi trường thực tế, đánh giá hiệu quả của giải pháp trong việc giải quyết các
  thách
  thức của DeFi, và đưa ra các kết luận cũng như hướng phát triển tiếp theo cho
  dự án.

  \textbf{Từ khóa}: \textit{LaunchCrypt, DeFi, Aptos, Solana,
    Avalanche, ReactJS, NestJS, Move, Solidity, Rust, Tradingview, Web 2.0}

  % \vspace*{1cm}
  % \textbf{Keywords}: \textit{SGP, SAT, SAT Encoding, CNF Encoding.}

\end{small}\newpage\cleardoublepage
% Abstract English
\begin{center}
  \textbf{\large{ABSTRACT}	}
\end{center}

\addcontentsline{toc}{chapter}{ABSTRACT}

\begin{small}

\textbf{Abstract:} 


\textbf{Keywords}: \textit{}

  % \vspace*{1cm}
  % \textbf{Keywords}: \textit{SGP, SAT, SAT Encoding, CNF Encoding.}
  % Khoá luận của tôi sẽ được sử dụng một công cụ tên USE (UML-based Specification Environment), công cụ này hỗ trợ đặc tả mô hình theo cú pháp của uml và ocl. Công cụ còn có một plugin là Model Validator. USE Model Validator sẽ hỗ trợ validate mô hình trên các ocl  constraints đã định nghĩa, nó sẽ trả về một object diagram nếu mô hình có thể thoả mãn (SATISFIABLE case) hoặc UNSATISFIABLE. Nhưng mô hình uml và ocl chỉ đặc tả các thuộc tính cấu trúc mà không thể hiện được thuộc tính thời gian (system execution path). Tôi sẽ sử dụng một plugin khác của USE để chuyển đổi mô hình ban đầu thành dạng filmstrip (là mô hình gốc ban đầu và thêm các class Snapshot, OperationCall) để thể hiện system execution path thành dạng sequence of snapshot.
  % tôi đang viết khoá luận của mình với đề tài: "A SUPPORT TOOL TO SPECIFY AND VERIFY TEMPORAL PROPERTIES IN OCL". Mục tiêu chính của khoá luận là mở rộng ngôn ngữ ocl để hỗ trợ đặc tả thuộc tính thời gian (temporal properties) và sự kiện (event) từ đó kiểm tra tính đúng đắn của mô hình trong kỹ thuật model-driven engineering. Khoá luận của tôi sẽ được sử dụng một công cụ tên USE (UML-based Specification Environment), công cụ này hỗ trợ đặc tả mô hình theo cú pháp của uml và ocl. Công cụ còn có một plugin là Model Validator. USE Model Validator sẽ hỗ trợ validate mô hình trên các ocl  constraints đã định nghĩa, nó sẽ trả về một object diagram nếu mô hình có thể thoả mãn (SATISFIABLE case) hoặc UNSATISFIABLE. Nhưng mô hình uml và ocl chỉ đặc tả các thuộc tính cấu trúc mà không thể hiện được thuộc tính thời gian (system execution path). Tôi sẽ sử dụng một plugin khác của USE để chuyển đổi mô hình ban đầu thành dạng filmstrip (là mô hình gốc ban đầu và thêm các class Snapshot, OperationCall) để thể hiện system execution path thành dạng sequence of snapshot.. tôi muốn bạn đóng vai một giáo sư giúp tôi cấu trúc khoá luận của tôi. Hiện tại tôi đang dự kiến khoá luận sẽ có 5 phần chính. Phần 1 introduction sẽ giới thiệu về bài toán xoay quanh model-driven engineering và việc đảm bảo tính đúng đắn của mô hình và việc validate chúng. Nhưng hiện tại OCL có những giới hạn về việc đặc tả các thuộc tính thời gian và event. Phần 2: Foundational Knowledge sẽ tập trung và các kiến thức nền tảng (kiến thức về mde, ocl, USE tool, Filmstrip model, antlr4, TOCL operators, events trong OOP paradigm, ...). Phần 3: OCL Extension sẽ là phần trình bày về phương pháp mà tôi đã sử dụng (phần đóng góp chính). Khoá luận của tôi sử dụng TOCL operators (always P, sometime P, sometime P before Q, next P, previous P, ...) là kết quả của một bài báo khác, họ áp dụng cho mô hình STM (Snpashot Transition Model) (Khác với Filmstrip của tôi một chút). Tôi có tận dụng kết quả đó và điều chỉnh cho mô hình Filmstrip của tôi, bên cạnh đó  tôi còn mở rộng định nghĩa event cho ocl (isCalled(operation) =  operation (call/start/end) events in OOP paradigm, becomesTrue(P) = state change events in OOP paradigm). Phần 4: Implementation and Experiments, tôi thực hiện phương pháp đưa ra bằng việc phát triển một plugin của công cụ USE. Plugin này sẽ thực hiện việc chuyển đổi TOCL (định nghĩa dựa trên UML&OCL Application model) sang OCL thông thường (Định nghĩa trên Filmstrip model). Phần 5 là kết luận.  
\end{small}
  % Khóa luận tập trung nghiên cứu và phát triển ứng dụng
  %   LaunchCrypt,
  %   một nền tảng đổi mới trong lĩnh vực tài chính phi tập trung (DeFi), nhằm đơn
  %   giản hóa
  %   quá trình tạo và giao dịch token cho người dùng không có kiến thức chuyên sâu
  %   về
  %   công nghệ. Để xây dựng nền tảng này, khóa luận đề xuất sử dụng stack công
  %   nghệ hiện
  %   đại bao gồm ReactJS cho phần frontend, NestJS cho backend, và các ngôn ngữ
  %   lập
  %   trình hợp đồng thông minh như Move, Solidity, và Rust để hỗ trợ đa chuỗi. Bài toán
  %   chính mà
  %   khóa luận giải quyết là tạo ra một hệ sinh thái toàn diện cho việc tạo, giao
  %   dịch, và
  %   phát triển token một cách an toàn và hiệu quả. Để có thể hiểu và áp dụng đúng
  %   các
  %   công nghệ cho bài toán này, khóa luận trước tiên đã nêu một số kiến thức nền
  %   tảng về
  %   công nghệ chuỗi khối, hợp đồng thông minh, các nguyên lý hoạt động của DeFi, cũng như các
  %   giao thức
  %   và chuẩn token trên các chuỗi khối khác nhau. Sau khi tìm hiểu tất cả các
  %   kiến thức
  %   nền tảng liên quan, khóa luận vận dụng phát triển ba mô đun chính của
  %   LaunchCrypt:
  %   (1) mô đun tạo token, cho phép người dùng dễ dàng tạo ra token của riêng họ
  %   trên các
  %   chuỗi được hỗ trợ, (2) mô đun giao dịch token, tự động tạo cặp giao dịch
  %   và cung
  %   cấp giao diện cho việc mua bán token, và (3) mô đun tốt nghiệp token, quản lý
  %   quá
  %   trình chuyển token lên các sàn giao dịch phi tập trung lớn hơn khi đạt đủ
  %   thanh khoản.
  %   Bên cạnh đó, khóa luận cũng phát triển các tính năng bổ sung như stake token,
  %   khả năng tương tác, kết nối giữa các đối tượng người dùng cùng với khả năng hỗ
  %   trợ đa chuỗi, bao gồm Aptos, Solana và Avalanche. Điều này không chỉ mở rộng
  %   phạm vi tiếp cận của dự án mà còn thúc đẩy tính linh hoạt và khả năng tương tác
  %   giữa các chuỗi khối khác nhau. Bằng
  %   cách cung cấp các công cụ như giao diện Tradingview cho giao dịch nâng cao,
  %   tính năng
  %   mua/bán tự động và thông tin chi tiết về token, LaunchCrypt không chỉ phục vụ
  %   người
  %   dùng mới mà còn đáp ứng nhu cầu của các nhà giao dịch có kinh nghiệm. Hơn
  %   nữa,
  %   việc tích hợp với các giải pháp để hỗ trợ người dùng Web 2.0 thể hiện tầm
  %   nhìn dài hạn
  %   của dự án trong việc thu hẹp khoảng cách giữa tài chính truyền thống và DeFi.
  %   Cuối
  %   cùng, khóa luận sẽ đề cập cách triển khai và thử nghiệm nền tảng LaunchCrypt
  %   trên
  %   môi trường thực tế, đánh giá hiệu quả của giải pháp trong việc giải quyết các
  %   thách
  %   thức của DeFi, và đưa ra các kết luận cũng như hướng phát triển tiếp theo cho
  %   dự án.\newpage\cleardoublepage
% Assurance
\setlength{\parindent}{1cm}

\begin{center}
  \textbf{\large{DECLARATION}	}
\end{center}
\addcontentsline{toc}{chapter}{DECLARATION}

I hereby declare that I composed this thesis, 
\textbf{\textit{"A Support Tool to Specify and Verify Temporal Properties in OCL"}},
under the supervision of Assoc. Prof. Dang Duc Hanh. 
This work reflects my own effort and serious commitment to research.
I have incorporated and adapted select open-source code and modeling resources to align with the research objectives, and all external materials used have been properly cited.
I take full responsibility for the content and integrity of this thesis.

\vspace{1cm}
\begin{flushright}
  \begin{minipage}{8cm}
    \centering
    \textit{Ha Noi, 07th May 2025}\\[0.2cm] % TODO: Change date
    \textbf{Student}\\[2.5cm]

    \textbf{Dinh Minh Hai}
  \end{minipage}
\end{flushright}\newpage\cleardoublepage
% Acknowledgement
\setlength{\parindent}{1cm}
\setcounter{page}{1}
\pagenumbering{roman}

\begin{center}
  \textbf{\large{LỜI CẢM ƠN}	}
\end{center}
\addcontentsline{toc}{chapter}{LỜI CẢM ƠN}

Được sự phân công của Khoa Công nghệ thông tin, Trường Đại học Công Nghệ -
ĐHQGHN và sự đồng ý của Thầy giáo hướng dẫn – PGS.TS. Đặng Đức Hạnh, em đã hoàn
thành đề tài: \textbf{\textit{"Phát triển hệ thống tạo lập và quản lý giao dịch
    đồng tiền điện tử phi tập trung trên nền tảng đa chuỗi khối"}}

Để hoàn thành khóa luận này, em xin chân thành cảm ơn thầy giáo hướng dẫn –
PGS.TS. Đặng Đức Hạnh đã dìu dắt, tận tình hướng dẫn em cả trên phòng nghiên
cứu và trong suốt khóa luận.

Em cũng xin cảm ơn các anh chị cựu sinh viên khóa trước, các bạn trong nhóm
nghiên cứu đã cùng nhau giúp đỡ vào thảo luận cùng em trong quá trình nghiên
cứu đề
tài.

Xin cảm ơn thầy cô Trường Đại học Công Nghệ đã giảng dạy tận tâm, nhiệt tình,
chính điều đó đã giúp em có được những kiến thức nền tảng quý báu.

Sau cùng, em gửi lời biết ơn đến gia đình, người nuôi dưỡng, tạo điều kiện và
động
lực cho em học tập và yên tâm hoàn thành khóa luận này.

Dù đã cố gắng thực hiện đề tài một cách tốt nhất thế nhưng do sự hạn chế về kiến thức và
kinh nghiệm,
khóa luận này của em không thể tránh khỏi những thiếu sót mà bản thân em chưa
thấy
được. Rất mong nhận được những lời nhận xét, sự góp ý của các thầy cô cùng các
bạn sinh
viên để bài khóa luận của em được hoàn chỉnh hơn.

Em xin chân thành cảm ơn!
\newpage\cleardoublepage


\addcontentsline{toc}{chapter}{TABLE OF CONTENTS}
\renewcommand{\contentsname}{\centering \fontsize{22}{0}\selectfont \bfseries TABLE OF CONTENTS}
\tableofcontents\newpage\cleardoublepage
%

\newpage
\addcontentsline{toc}{chapter}{LIST OF FIGURES}
\renewcommand{\listfigurename}{\centering \fontsize{22}{0}\selectfont \bfseries LIST OF FIGURES}
\listoffigures\cleardoublepage

\newpage
\addcontentsline{toc}{chapter}{LIST OF TABLES}
\renewcommand{\listtablename}{\centering \fontsize{22}{0}\selectfont \bfseries LIST OF TABLES}
\listoftables

\newpage
\chapter*{GLOSARY OF ABBREVIATIONS AND TERMS}
\addcontentsline{toc}{chapter}{GLOSARY OF ABBREVIATIONS AND TERMS}

%\begin{tabular}{|l|l|l|l|}
\begin{longtable}{|>{\raggedright\arraybackslash}p{4cm}|>{\raggedright\arraybackslash}p{10cm}|}
  \hline
  \textbf{Acronyms}       & \textbf{Definition}           
  \\
  \hline
  MDE                    & Model Driven Engineering  
  \\
  \hline
  UML                    & Unified Modeling Language    
  \\
  \hline
  OCL                    & Object Constraint Language
  \\
  \hline
  USE                    & UML-based Specification Environment
  \\
  \hline
  DEX                    & Decentralized Exchange
  \\
  \hline
  SPL                    & Solana Program Library                
  \\
  \hline
  SDK                    & Software development kit
  \\
  \hline
  DOM                    & Document Object Model 
  \\
  \hline
\end{longtable}\newpage\cleardoublepage
%\listoftables

%
\setcounter{page}{1}

\pagenumbering{arabic}

\setlength{\parindent}{1cm}

\begin{center}
  \section*{INTRODUCTION}
\end{center}
\addcontentsline{toc}{chapter}{INTRODUCTION}

Modern software development faces significant challenges as
systems grow increasingly complex. Traditional development approaches relying on 
manual coding often struggle to manage this complexity, leading to higher error
rates and extended development cycles. These problems often come from the development 
process, not the system requirements. Model-Driven Engineering (MDE) \cite{MDE} 
addresses these challenges by shifting the focus from code to models. 
In MDE, developers use models to design systems, and tools can automatically generate 
code, documentation, and tests from those models. 
The Unified Modeling Language (UML) \cite{UML} and the Object 
Constraint Language (OCL) \cite{OCL} have become the \textit{de facto} standards 
for model-driven approaches. UML provides a rich set of visual modeling concepts to 
represent the structural and behavioral aspects of a system, while OCL allows 
specifying constraints and structural properties of UML models. However, for 
complex systems, it is often necessary to specify and verify dynamic behaviors 
that involve temporal constraints and event-driven conditions. Unfortunately, OCL 
lacks the expressiveness to model these dynamic aspects, which limits its ability 
to specify and verify temporal properties and event-based behaviors.

This thesis aims to address this limitation by extending OCL with constructs for
temporal properties and events, thereby enhancing its expressiveness in modeling dynamic 
system aspects. We implement this extension, called TOCL+, as a plugin for the 
UML-based Specification Environment (USE) \cite{USE}, a tool that supports the 
specification and validation of software systems using UML and OCL. To enable both
specification and verification of temporal properties, we employ a 
technique known as filmstripping \cite{Filmstripping}, which transforms models 
with dynamic temporal constraints into structurally equivalent models that can 
be analyzed using existing verification tools. Our plugin automatically translates 
temporal OCL expressions into standard OCL constraints on filmstrip models, 
allowing modelers to leverage the existing USE model validator \cite{USE_Validator} 
for verification. This approach bridges the gap between expressing temporal 
requirements and verifying them, providing a complete solution that integrates 
seamlessly with the established USE environment and its validation capabilities.
% This thesis aims to address this limitation by extending OCL with constructs for
% temporal properties and events, enhancing its expressiveness in modeling dynamic 
% system aspects. We implement this extension as a plugin, called TOCL+, for the 
% UML-based Specification Environment (USE) \cite{USE}, a tool that supports the 
% specification and validation of software systems using UML and OCL. To enable not 
% only specification but also verification of temporal properties, we employ a 
% technique known as filmstripping \cite{Filmstripping}, which transforms models 
% with dynamic temporal constraints into structurally equivalent models that can 
% be analyzed using existing verification tools. Our plugin automatically translates 
% temporal OCL expressions into standard OCL constraints on a filmstrip model, 
% allowing modelers to leverage the existing USE model validator \cite{USE_Validator} 
% for verification. This approach bridges the gap between expressing temporal 
% requirements and verifying them, providing a complete solution that integrates 
% seamlessly with the established USE environment and its validation capabilities.

The thesis is structured as follows:
\begin{itemize}
  \item \textbf{Chapter 1: Background} lays the foundation for understanding the theoretical concepts and tools used in this thesis.
  
  \item \textbf{Chapter 2: Specification and Verification of Temporal Properties in OCL} presents our approach to extending OCL for specifying and verifying temporal properties.

  \item \textbf{Chapter 3: Implementation and Experiment} describes the implementation of our approach and evaluates it through a case study.

  \item \textbf{Conclusion} summarizes the contributions of this thesis and discusses future work.
\end{itemize}\newpage\cleardoublepage
\chapter{FOUNDATIONAL KNOWLEDGE}

\setlength{\parindent}{1cm}

\section{Introduction}

\hspace{1cm} This chapter presents fundamentals about concepts and artifacts essential 
to this thesis. The modeling languages such as, Unified Modeling Language (UML), 
together with Object Constraint Language (OCL), are used to describe structural and 
behavioral aspects of systems and are briefly described in this chapter. A description 
of the modeling and specification tool called UML-based Specification Environment (USE)
is presented, including its model validation capabilities that form the foundation for 
our verification approach. We explain the filmstrip model transformation process in 
detail, as it serves as the underlying mechanism for our temporal verification approach.
\newpage\cleardoublepage
\chapter{Temporal and Event Constructs for OCL}

\section{Introduction}

\hspace{1cm} OCL provides strong support for structural properties in UML models but 
falls short when specifying dynamic system behavior. Operating only on single states 
or individual transitions, OCL cannot express properties spanning multiple states or 
responding to system events. This limitation is significant for modern systems 
requiring temporal and reactive behaviors.

Temporal logics like LTL \cite{LTL} and CTL offer formal frameworks for temporal properties but 
require specialized knowledge unfamiliar to most UML designers. This creates a practical 
barrier for practitioners comfortable with UML/OCL but not with formal temporal notations.

This chapter presents two main contributions:

First, we build upon the existing Temporal OCL (TOCL) extension \cite{TOCL}, which already 
incorporates temporal operators like \textit{always}, \textit{sometime}, and 
\textit{until} for reasoning about system evolution over time. However, TOCL remains 
limited in its ability to express event-based properties. Our TOCL+ extension 
addresses this limitation by introducing enhanced event constructs for detecting 
specific system occurrences such as operation calls and state changes, along with 
support for bounded existence properties (e.g., "an operation is called exactly n 
times" or "at most n times"). TOCL+ maintains OCL's familiar syntax while 
significantly expanding its capabilities for specifying event-driven and quantified 
temporal behaviors.
% First, TOCL+ extends OCL with temporal and event capabilities. It adds temporal 
% operators like \textit{always}, \textit{sometime}, and \textit{until} for reasoning 
% about system evolution over time, and introduces event constructs for detecting 
% specific system occurrences such as operation calls and state changes. TOCL+ maintains 
% OCL's familiar syntax while enabling complex dynamic specifications.

Second, we introduce an approach that enables verification of TOCL+ 
specifications using existing tools. This approach transforms UML/OCL models into 
filmstrip models representing state sequences, and translates TOCL+ specifications 
into standard OCL constraints verifiable within these models.

% Third: Implementation of TOCL+ to OCL transformation
Third, we present our implementation approach for transforming TOCL+ expressions into standard OCL constraints. This transformation defines practical translation patterns for temporal operators and event constructs, mapping them to structural navigations through snapshots and operation calls in the filmstrip model. These translations enable the verification of complex temporal properties using existing OCL verification tools without requiring specialized temporal logic model checkers.

The chapter is organized as follows:
\begin{itemize}
    \item Section 2.2 presents the specification approach.
    
    \item Section 2.3 details the verification approach.
    
    \item Section 2.4 describes the implementation of TOCL+ to OCL transformation.
\end{itemize}

Together, these contributions provide a complete solution for both specifying and 
verifying temporal properties within the model-driven engineering paradigm.
\input{chapters/c2/c2_event_extension}

\newpage\cleardoublepage
\chapter{Implementation and Experiment}

\section{Introduction}

\hspace{1cm} In the previous chapter, we introduced TOCL+, an extension of OCL with temporal operators and event constructs, and presented a theoretical framework for specifying and verifying temporal properties in UML models. While the formal semantics and verification approach provide a solid foundation, they require practical tool support to be effectively applied. This chapter bridges the gap between theory and practice by presenting our implementation of a TOCL+ plugin for the USE tool and demonstrating its effectiveness through a detailed case study.

The primary objective of this chapter is twofold. First, we describe the implementation of our TOCL+ plugin for the USE tool, which enables modelers to specify temporal properties using our extended language and automatically verify them using the filmstrip approach. Second, we demonstrate the practical application of TOCL+ through a detailed case study of the Software System model introduced in Chapter 2, showcasing how various types of temporal properties can be effectively specified and verified.

In the following sections, we present the architecture and implementation of our plugin, focusing on how it integrates with the USE tool environment and implements the transformation process described in Chapter 2. We then explore the Software System case study, illustrating how our approach handles the specification and verification of different types of temporal properties in practice.
\section{TOCL+ Tool Support and Implementation}

\subsection{Support tool architecture}

\begin{figure}
    \centering
    \includegraphics[width=1\textwidth]{figures/c3/Architecture_overview.png}
    \caption{Support Tool Architecture and Verification Workflow.}
    \label{sec:plugin_support_tool_architecture}
\end{figure}

\hspace{1cm} Figure \ref{sec:plugin_support_tool_architecture} illustrates the 
architecture and workflow of our TOCL+ support tool. This integrated verification 
framework combines several components: the existing USE environment, the Filmstrip 
plugin, our TOCL+ plugin, and the Model Validator. The architecture maintains a 
clear separation between modeling, specification, and verification concerns while 
providing a cohesive workflow for users.

The verification process consists of five distinct steps, beginning with preparation 
and ending with validation. First, the user prepares two input files: a standard 
\texttt{.use} file containing the UML/OCL application model and a \texttt{.tocl} 
file containing TOCL+ property specifications. Second, the user loads the application 
model into USE to make it available for transformation. Third, the user activates 
our plugin through the USE interface, selecting both a destination path for the 
output model and the \texttt{.tocl} file containing the temporal properties to verify.

Internally, the plugin then executes a two-phase transformation process. In the 
first phase, it invokes the Filmstrip plugin to transform the application model 
into a filmstrip model following the rules described in Section~\ref{subsec:filmstripping}. 
In the second phase, it processes the TOCL+ expressions using our ANTLR4-generated 
parser and listener components, which implement the transformation rules for 
converting temporal specifications into equivalent OCL constraints. These generated 
constraints are added to the list of invariants in the output model file alongside 
the filmstrip model elements.

To complete the verification process, the user loads this output model back into 
USE together with a configuration file that establishes search bounds, and then 
employs the Model Validator to analyze the constraints. The validator systematically 
explores the search space, determining whether the temporal properties are satisfied 
and providing a model instance as evidence when applicable.

Our primary contribution in this architecture is the TOCL+ plugin, specifically 
the TOCL+ to OCL transformation component that enables the verification of temporal 
properties using existing OCL tools. The implementation details of this transformation 
process, including the translation rules for different temporal operators and event 
constructs, are presented in the next subsection.

\subsection{Implementation of TOCL+ to OCL Transformation}

\hspace{1cm} The core of our contribution lies in the transformation of TOCL+ 
expressions into equivalent OCL constraints that can be verified over filmstrip 
models. This transformation process involves systematically mapping temporal operators 
and event constructs to structural navigations through snapshots and operation calls. 
In this section, we describe our implementation approach and the key translation 
patterns we developed.

Our transformation approach is inspired by the work of \cite{TOCL2OCL}, who 
transformed TOCL \cite{TOCL} into OCL in the context of a Snapshot-Transition Model 
(STM). While their approach also converts temporal properties into static ones, we 
adapted and extended it to work within the filmstrip model context.

To transform TOCL+ expressions, we defined translations for TOCL+ operators and 
events to OCL, as shown in Table \ref{tab:TOCL2OCL}. To create these translations, 
we utilized several query operations provided by the filmstrip structure. 
The self.snapshot query accesses the snapshot associated with an object in the 
"current state" where the expression is being evaluated. The pred() and succ() 
operations, when applied to a snapshot, navigate to the previous and next state 
respectively. For objects to navigate to their corresponding versions in adjacent 
states, we use the two associations .pred and .succ. These navigation mechanisms 
form the foundation for implementing our temporal operator translations.

Note that filmstrip model does not inherently provide any means to identify the same 
logical object across different states - it only provides the .pred and .succ 
associations to navigate between corresponding objects in adjacent snapshots. 
In order to overcome this limitation, we require modelers to add an id attribute 
to all classes in the application model. As seen in Figure \ref{fig:object_diagram_liveness}, 
this allows us to identify the same logical object across different snapshots. 
Internally, when the TOCL+ plugin transforms the model, we add additional 
constraints to ensure this id remains consistent between different states. 
This id attribute is critical in the OCL translation, particularly for event 
constructs like becomesTrue. When navigating with expressions like 
self.snapshot.pred().[ContextObject], we get a collection of objects of the same 
type, and we select the one with matching identity using ->any(o | o.id = self.id).

Table \ref{tab:TOCL2OCL} presents the complete set of translation patterns we 
developed for TOCL+ operators and event constructs. Each pattern systematically 
maps a TOCL+ construct to an equivalent OCL expression interpreted over the filmstrip 
model. The patterns use placeholders (indicated by square brackets) that get 
substituted during the transformation process: \texttt{[s |= P]} indicates that 
property P holds in snapshot s; \texttt{[ContextSnapshot]} is the snapshot in which 
the property is evaluated; \texttt{[ContextObject]} is the object on which the 
property is evaluated; and \texttt{[OpClassName]} is the class representing an 
operation call in the filmstrip model. When applying these patterns, each placeholder 
is replaced with the appropriate expression based on the model context. For example, 
in the liveness property translation shown in Listing \ref{lst:ocl_liveness}, the 
\texttt{[ContextSnapshot]} is replaced with the local variable \texttt{s} inside the 
\texttt{exists} query. The bounded quantifiers (e.g., \texttt{at most}, \texttt{at least}) 
are translated into corresponding comparators (\texttt{<=}, \texttt{>=}). 
The bounded quantifiers in TOCL+ expressions are systematically translated into their 
corresponding OCL comparators: \texttt{at most} becomes less than or equal to 
(\texttt{<=}), while \texttt{at least} becomes greater than or equal to (\texttt{>=}). 
For bounded existence properties without an explicit quantifier (e.g., 
\texttt{isCalled(Op()) 3 times}), the translation applies the equality operator 
(\texttt{=}), enforcing that the event occurs exactly the specified number of times. 
In contrast, when dealing with unbounded event expressions without quantifiers (e.g., 
simply \texttt{isCalled(Op())}), the translation converts the \texttt{select} operation 
into an \texttt{exists} operation, requiring only that the event occurs at least once 
rather than counting occurrences.
    
We implemented the transformation using ANTLR4, a parser generator that creates a 
parse tree from TOCL+ expressions. After defining the translation patterns shown 
in Table \ref{tab:TOCL2OCL}, we created Java listener classes that extend the 
generated parser listeners. The transformation process employs these listener 
components to traverse the parse tree and produce corresponding OCL expressions 
by overriding the generated listener methods. As the parser walks through each node 
in the parse tree, our listeners intercept parse tree events and apply the appropriate 
translation rules, constructing equivalent OCL constraints that navigate through 
the filmstrip structure. The transformation follows a consistent pattern: when a 
temporal operator or event construct is encountered, the listener extracts relevant 
information from the parse tree nodes, applies the corresponding translation pattern, 
and builds the equivalent OCL expression. Our implementation maintains a stack of 
expressions to handle nested structures, pushing both the original TOCL+ expression 
and its OCL translation for later integration into the output model.
Listing \ref{lst:becomesTrue_translation} provides an example of this process for 
the \texttt{becomesTrue} event construct, showing how we extract the expression 
to be evaluated, establish the necessary context, and construct the translated OCL 
expression. 

After processing all the nodes in the parse tree, our implementation finalizes the 
translation in the root node visit. Since TOCL+ is an extension of OCL, many 
constructs remain unchanged and are directly preserved during translation. The 
completion process begins by accessing the token stream to retrieve the complete 
original expression text. Our implementation then systematically pops each translated 
OCL fragment and its corresponding original TOCL+ expression from the stack. Using 
string replacement operations, it substitutes each temporal construct with its 
equivalent OCL translation while preserving the structure of the original expression. 
This approach allows us to handle nested temporal operators naturally, as inner 
expressions are processed before their containing expressions. The final translated 
OCL constraint is then attached to the root of the parse tree and ultimately added 
to the output model as an invariant. This complete process ensures that complex 
temporal properties are accurately transformed into standard OCL constraints that 
can be verified by the Model Validator.

\begin{lstlisting}[
    style=javastyle,
    caption={Translation of becomesTrue event to OCL.},
    label={lst:becomesTrue_translation}
]
TokenStream tokens = parser.getTokenStream();
String originalEvent = tokens.getText(ctx);
String translatedEvent;
// P
String expressionToSatisfy = getOCL(ctx.getChild(2)); 
// e.g., "system", "application"
String roleName = toLowerFirstChar(currentContext); 
String currentSnapshot = "self.snapshot"; 
String selectObject = "->any(o | o.id = self.id)";

// e.g. self.snapshot.system->any(o | o.id = self.id)
String objectAtCurrentSnapshot = currentSnapshot + "." + roleName + selectObject;
String objectAtPreviousSnapshot = currentSnapshot + ".pred()." + roleName + selectObject;
String P_at_currentSnapshot = expressionToSatisfy.replace("self", "currentObject");
String P_at_previousSnapshot = expressionToSatisfy.replace("self", "previousObject");

translatedEvent = 
"let currentObject = " + objectAtCurrentSnapshot +
" in let previousObject = " + objectAtPreviousSnapshot +
" in not (" + P_at_previousSnapshot + ") and (" + P_at_currentSnapshot + ")";

eventStack.push(translatedEvent);
eventStack.push(originalEvent);    
\end{lstlisting}

\begin{longtable}{|>{\footnotesize}p{0.6cm}|>{\scriptsize\raggedright\arraybackslash}p{4cm}|>{\scriptsize\raggedright\arraybackslash}p{\dimexpr\textwidth-4.6cm-4\tabcolsep-3\arrayrulewidth\relax}|}
    \caption{Translation of TOCL+ operators and events to OCL.}
    \label{tab:TOCL2OCL} \\
    \hline
    \textbf{No.} & \textbf{TOCL+} & \textbf{OCL Translation} \\
    \hline
    1 & 
    next P &
    let nextSnapshot:Snapshot = self.snapshot.succ() in [nextSnapshot |= P] \\
    \hline
    2 &
    always P &
    let CS:Snapshot = self.snapshot in Set\{CS\}->closure(s | s.succ())->forAll(s | [s |= P]) \\
    \hline  
    3 &
    always P until Q &
    let CS:Snapshot = self.snapshot
    in let FS:Set(Snapshot) = Set\{CS.succ()\}->closure(s | s.succ())
    in let AllFSQ:Set(Snapshot) = FS->select(s | [s |= Q])
    in let FSQ:Snapshot = AllFSQ->any(s | Set\{s\}->closure(s | s.succ())->includesAll(AllFSQ))
    in let afterQ:Set(Snapshot) = Set\{FSQ\}->closure(s | s.succ())
    in let FSP:Set(Snapshot) = FS->select(s | [s |= P])
    in if FSQ.isDefined() then (if (FSP->size() > 0) then (FS-afterQ = FSP-afterQ) else false endif) else (FS = FSP) endif \\
    \hline
    4 &
    always P since Q &
    let CS:Snapshot = self.snapshot
    in let PS:Set(Snapshot) = Set\{CS.pred()\}->closure(s | s.pred())
    in let AllPSQ:Set(Snapshot) = PS->select(s | [s |= Q])
    in let PSQ:Snapshot = AllPSQ->any(s | Set\{s\}->closure(s | s.pred())->includesAll(AllPSQ))
    in let beforeQ:Set(Snapshot) = Set\{PSQ\}->closure(s | s.pred())
    in let PSP:Set(Snapshot) = PS->including(CS)->select(s | [s |= P])
    in if PSQ.isDefined() then (if (PSP->size() > 0) then (PS->including(CS)-beforeQ = PSP-beforeQ) else false endif) else (PSP = PS->including(CS)) endif \\
    \hline
    5 &
    sometime P &
    let CS:Snapshot = self.snapshot in Set\{CS\}->closure(s | s.succ())->exists(s | [s |= P]) \\
    \hline
    6 &
    sometime P before Q &
    let FS:Set(Snapshot) = Set\{self.snapshot\}->closure(s | s.succ())
    in let PreS:Set(Snapshot) = Set\{self.snapshot.pred()\}->closure(s | s.pred())
    in let AllFSQ:Set(Snapshot) = FS->select(s | [s |= Q])
    in let FSQ:Snapshot = AllFSQ->any(s | Set\{s\}->closure(s | s.succ())->includesAll(AllFSQ))
    in let FSP:Set(Snapshot) = FS->select(s | [s |= P])
    in if FSQ.isDefined() then (if (FSP->size() > 0) then ((Set\{FSQ.pred()\}->closure(s | s.pred())-PreS)->exists(s\_1 | FSP->includes(s\_1))) else false endif) else false endif \\
    \hline
    7 &
    sometime P since Q &
    let CS:Snapshot = self.snapshot
    in let PS:Set(Snapshot) = Set\{CS.pred()\}->closure(s | s.pred())
    in let AllPSQ:Set(Snapshot) = PS->select(s | [s |= Q])
    in let PSQ:Snapshot = AllPSQ->any(s | Set\{s\}->closure(s | s.pred())->includesAll(AllPSQ))
    in let PSP:Set(Snapshot) = PS->select(s | [s |= P])
    in if PSQ.isDefined() then (Set{PSQ}->closure(s | s.succ())->excluding(null)->intersection(PS)->exists(s | PSP->includes(s))) else false endif \\ 
    \hline
    8 &
    previous P &
    let previousSnapshot:Snapshot = self.snapshot.pred() in [previousSnapshot |= P] \\
    \hline
    9 &
    sometimePast P &
    let CS:Snapshot = self.snapshot in Set\{CS.pred()\}->closure(s | s.pred())->exists(s | [s |= P]) \\
    \hline
    10 &
    alwaysPast P &
    let CS:Snapshot = self.snapshot in Set\{CS.pred()\}->closure(s | s.pred())->forAll(s | [s |= P]) \\
    \hline
    11 &
    isCalled(Op()) &
    [OpClassName].allInstances()->exists(op | op.succ() = [ContextSnapshot]) \\
    \hline
    12 &
    isCalled(Op($param_1$, $\ldots$, $param_n$)) &
    [OpClassName].allInstances()->exists(op | op.succ() = [ContextSnapshot] and
    (Set\{op.$param_1$.succ\}->closure(p | p.succ)->includes($param_1$) or Set\{op.$param_1$.pred\}->closure(p | p.pred)->includes($param_1$))
    and ($\ldots$)
    and (Set\{op.$param_n$.succ\}->closure(p | p.succ)->includes($param_n$) or Set\{op.$param_n$.pred\}->closure(p | p.pred)->includes($param_n$))) \\
    \hline
    13 &
    isCalled(Op()) [at most | at least | ] n times &
    [OpClassName].allInstances()->select(op | op.succ() = [ContextSnapshot])->size() [<= | >= | =] n \\
    \hline
    14 &
    isCalled(Op($param_1$, $\ldots$, $param_n$)) [at most | at least | ] n times &
    [OpClassName].allInstances()->select(op | op.succ() = [ContextSnapshot] and
    (Set\{op.$param_1$.succ\}->closure(p | p.succ)->includes($param_1$) or Set\{op.$param_1$.pred\}->closure(p | p.pred)->includes($param_1$)) 
    and ($\ldots$) 
    and (Set\{op.$param_n$.succ\}->closure(p | p.succ)->includes($param_n$) or Set\{op.$param_n$.pred\}->closure(p | p.pred)->includes($param_n$)))->size() [<= | >= | =] n \\
    \hline
    15 &
    becomesTrue(P) &
    let currentObject = self.snapshot.[ContextObject]->any(o | o.id = self.id) in
    let previousObject = self.snapshot.pred().[ContextObject]->any(o | o.id = self.id) in 
    not [previousObject |= P] and [currentObject |= P] \\
    \hline
\end{longtable}

\section{Case study: Software System model}
\label{sec:case_study_software_system}

\subsection{Model Specification}

\hspace{1cm} The Software System model shown in Listing \ref{lst:software_system_specification} 
is the same model introduced in Chapter 1 to demonstrate OCL constraints (see 
Figure \ref{fig:class_diagram_software_system}). As a reminder, this model represents 
a simplified operating system that manages applications through their lifecycle. 
It consists of two main classes: \texttt{System} and \texttt{Application}, connected 
through an association. The \texttt{System} maintains three collections 
(\texttt{loadedApps}, \texttt{installedApps}, and \texttt{runningApps}) representing 
different application states, and provides four operations to manage applications: 
\texttt{load}, \texttt{install}, \texttt{run}, and \texttt{stop}.

The \texttt{freeMemory} attribute represents available disk space, which decreases 
when applications are loaded. The \texttt{load} operation acts as a download action, 
reducing available memory when an application is acquired. The \texttt{install} 
operation processes all loaded applications at once, moving them from 
\texttt{loadedApps} to \texttt{installedApps}. When an application is running, it 
exists in both the \texttt{installedApps} and \texttt{runningApps} sets simultaneously. 
The \texttt{SystemApplication} association facilitates navigation between system and 
applications in this simplified model.

Note that the numerous postconditions like \texttt{sameInstalledAndRunning}, 
\texttt{sameRunning}, \texttt{sameMemory}, \texttt{sameLoaded}, \texttt{sameInstalled}, 
and the \texttt{unchanged} constraints serve as helper constraints in the verification 
context: since the Model Validator assigns random values to attributes when exploring 
possible states, these constraints ensure that attributes unaffected by an operation 
remain consistent between snapshots.

This model serves as an ideal case study for temporal property verification as it 
involves operations with clear sequential dependencies and state transitions that 
cannot be adequately expressed using standard OCL. The complete specification, 
including all constraints and operation contracts necessary for our verification 
experiments, is provided in Appendix A. This appendix contains the full USE model 
specification that forms the foundation for our temporal property verification.
% This model serves as an ideal case study for temporal property verification as it 
% involves operations with clear sequential dependencies and state transitions that 
% cannot be adequately expressed using standard OCL. The full specification below 
% includes all constraints and operation contracts necessary for our verification 
% experiments.

% \begin{lstlisting}[
%     caption={Specification of the Software System model in USE environment.},
%     label={lst:software_system_specification}
% ]
% model SoftwareSystem
% -- Classes
% class System
% attributes
%     id : Integer
%     freeMemory : Integer init = 10
%     loadedApps : Set(Application) init = Set{}
%     installedApps : Set(Application) init = Set{}
%     runningApps : Set(Application) init = Set{}
% operations
%     load(app : Application)
%     begin
%         self.loadedApps := self.loadedApps->including(app);
%         self.freeMemory := self.freeMemory - app.size;
%     end
%     install()
%     begin
%         self.installedApps := self.installedApps->union(self.loadedApps);
%         self.loadedApps := self.loadedApps->reject(true)->excluding(null);
%     end
%     run(app : Application)
%     begin
%         self.runningApps := self.runningApps->including(app);
%     end
%     stop(app : Application)
%     begin
%         self.runningApps := self.runningApps->excluding(app);
%     end
% end
% class Application
% attributes
%     id : Integer
%     size : Integer
% end
% -- Associations
% association SystemApplication between
%     System[1] role system
%     Application[0..*] role apps
% end
% -- Invariants
% constraints
% context System
%     inv memoryConstraint: self.freeMemory >= 0
%     inv notLoadedAndInstalled: self.loadedApps->intersection(self.installedApps)->isEmpty()
%     inv sets: let appNumber: Integer = self.apps->size() in
%         (self.loadedApps->size() <= appNumber and
%         self.installedApps->size() <= appNumber and
%         self.runningApps->size() <= appNumber)
%     inv notContainsNull:
%         not self.loadedApps->includes(null) and
%         not self.installedApps->includes(null) and
%         not self.runningApps->includes(null)

% context Application
%     inv sizeConstraint: self.size > 0

% context System::load(app: Application)
%     pre notLoaded: not self.loadedApps->includes(app) and
%                     not self.installedApps->includes(app) and
%                     not self.runningApps->includes(app)
%     pre enoughMemory: self.freeMemory >= app.size
%     post loaded: self.loadedApps = self.loadedApps@pre->including(app)
%     post freeMemory: self.freeMemory = self.freeMemory@pre - app.size
%     post unchanged:
%         self.apps->forAll(app |
%             app.size = app.size@pre and
%             app.id = app.id@pre)
%     post sameInstalledAndRunning:
%         self.installedApps = self.installedApps@pre and
%         self.runningApps = self.runningApps@pre

% context System::install()
%     pre hasLoadedApps: self.loadedApps->notEmpty()
%     post installed: self.installedApps = self.installedApps@pre->union(self.loadedApps@pre)
%     post loadedAppsEmpty: self.loadedApps = self.loadedApps@pre->reject(true)->excluding(null)
%     post sameRunning: self.runningApps = self.runningApps@pre
%     post sameMemory: self.freeMemory = self.freeMemory@pre
%     post unchanged:
%         self.apps->forAll(app |
%             app.size = app.size@pre and
%             app.id = app.id@pre)

% context System::run(app : Application)
%     pre isInstalled: self.installedApps->includes(app)
%     pre notRunning: not self.runningApps->includes(app)
%     post running: self.runningApps = self.runningApps@pre->including(app)
%     post sameLoaded: self.loadedApps = self.loadedApps@pre
%     post sameInstalled: self.installedApps = self.installedApps@pre
%     post sameMemory: self.freeMemory = self.freeMemory@pre
%     post unchanged:
%         self.apps->forAll(app |
%             app.size = app.size@pre and
%             app.id = app.id@pre)

% context System::stop(app : Application)
%     pre isRunning: self.runningApps->includes(app)
%             and self.installedApps->includes(app)
%     post notRunning: self.runningApps = self.runningApps@pre->excluding(app)
%     post sameInstalled: self.installedApps = self.installedApps@pre
%     post sameLoaded: self.loadedApps = self.loadedApps@pre
%     post sameMemory: self.freeMemory = self.freeMemory@pre
%     post unchanged:
%         self.apps->forAll(app |
%             app.size = app.size@pre and
%             app.id = app.id@pre)
% \end{lstlisting}


\subsection{Temporal Property Verification}
% For each of the 4 properties:
%   1. Brief description of what the property verifies
%   2. TOCL+ specification
%   3. Generated OCL translation
%   4. Verification results

% The TOCL+ temporal properties for the Software System model were shown in Listing 
% \ref{lst:tocl+}. And Listing \ref{lst:ocl_translations} shows the OCL translations 
% for those TOCL+ properties. 

% \hspace{1cm} To demonstrate our TOCL+ to OCL transformation approach, we verified four temporal 
% properties from Chapter 2 against the Software System model: safety1 (applications 
% must be loaded before being run), safety2 (applications must follow the 
% load-install-run sequence), safety3 (applications are loaded at most once), and 
% liveness (loaded applications will eventually be installed). Additionally, we 
% also add another version of the liveness property specified using becomesTrue event
% that have been specified in \ref{lst:liveness_becomesTrue}.
% \hspace{1cm} To demonstrate our TOCL+ to OCL transformation approach, we verified four temporal 
% properties from Chapter 2 against the Software System model: safety1 (applications 
% must be loaded before being run), safety2 (applications must follow the 
% load-install-run sequence), safety3 (applications are loaded at most once), and 
% liveness (loaded applications will eventually be installed). Listing \ref{lst:tocl+} 
% shows TOCL+ specifications for the above properties. We also included an 
% alternative specification of the liveness property using the \texttt{becomesTrue} event 
% construct (shown in Listing \ref{lst:liveness_becomesTrue}), demonstrating TOCL+'s 
% flexibility in expressing equivalent temporal properties through different constructs.
\hspace{1cm} To demonstrate the effectiveness of our TOCL+ to OCL transformation approach, we verified four temporal 
properties against the Software System model: safety1 (applications must be loaded before being run), 
safety2 (applications must follow the load-install-run sequence), safety3 (applications can be loaded at most once), 
and liveness (loaded applications eventually become installed). These properties, originally introduced in Chapter 2, 
exercise different aspects of TOCL+'s expressiveness. Listing \ref{lst:tocl+} presents the complete TOCL+ 
specifications. Additionally, we developed an alternative formulation of the liveness property using the 
\texttt{becomesTrue} event construct (Listing \ref{lst:liveness_becomesTrue}), illustrating TOCL+'s 
flexibility in expressing semantically equivalent properties through different syntactic constructs.

Listing \ref{lst:ocl_translations} shows 
the OCL constraints generated by our transformation plugin for these properties. 
While the resulting OCL expressions are considerably more complex than their TOCL+ 
counterparts—involving extensive navigation through the filmstrip model—they 
correctly implement the required temporal semantics. This demonstrates how our 
approach enables temporal reasoning within standard OCL, shielding modelers 
from the underlying complexity.

\begin{lstlisting}[
    caption={OCL Translations for TOCL+ properties shown in listing \ref{lst:tocl+}},
    label={lst:ocl_translations}
]
context System
inv safety1:
  self.runningApps->notEmpty() implies
  self.runningApps->forAll(app |
    (run_SystemOpC.allInstances()->exists(op | 
      op.succ() = self.snapshotSystem and 
      (Set{op.app.succApplication}->closure(p | p.succApplication)->includes(app) 
       or 
       Set{op.app.predApplication}->closure(p | p.predApplication)->includes(app))
    )) 
    implies
    (let CS:Snapshot = self.snapshotSystem in Set{CS.pred()}->closure(s | s.pred())->exists(s | (load_SystemOpC.allInstances()->exists(op | op.succ() = s and (Set{op.app.succApplication}->closure(p | p.succApplication)->includes(app) or Set{op.app.predApplication}->closure(p | p.predApplication)->includes(app))))))
  )

context System
inv safety2:
    self.runningApps->notEmpty() implies
    self.runningApps->forAll(app |
        (run_SystemOpC.allInstances()->exists(op | op.succ() = self.snapshotSystem and (Set{op.app.succApplication}->closure(p | p.succApplication)->includes(app) or Set{op.app.predApplication}->closure(p | p.predApplication)->includes(app)))) implies
        (let CS:Snapshot = self.snapshotSystem in let PS:Set(Snapshot) = Set{CS.pred()}->closure(s | s.pred())->excluding(null) in let AllPSQ:Set(Snapshot) = PS->select(s | (load_SystemOpC.allInstances()->exists(op | op.succ() = s and (Set{op.app.succApplication}->closure(p | p.succApplication)->includes(app) or Set{op.app.predApplication}->closure(p | p.predApplication)->includes(app))))) in let PSQ:Snapshot = AllPSQ->any(s | Set{s}->closure(s | s.pred())->includesAll(AllPSQ)) in let PSP:Set(Snapshot) = PS->select(s | install_SystemOpC.allInstances()->exists(op | op.succ() = s)) in if PSQ.isDefined() then (Set{PSQ}->closure(s | s.succ())->excluding(null)->intersection(PS)->exists(s | PSP->includes(s))) else false endif)
    )

context System
inv safety3:
    self.installedApps->notEmpty() implies
    self.installedApps->forAll(app |
        (let CS:Snapshot = self.snapshotSystem in Set{CS.pred()}->closure(s | s.pred())->exists(s | (load_SystemOpC.allInstances()->select(op | op.succ() = s and (Set{op.app.succApplication}->closure(p | p.succApplication)->includes(app) or Set{op.app.predApplication}->closure(p | p.predApplication)->includes(app)))->size() <= 1)))
    )

context System
inv liveness:
    self.loadedApps->notEmpty() implies
    self.loadedApps->forAll(app |
        (let CS:Snapshot = self.snapshotSystem in Set{CS}->closure(s | s.succ())->excluding(null)->exists(s | install_SystemOpC.allInstances()->exists(op | op.succ() = s)))
    )

context Application
inv livenessBecomesTrue:
    self.system.loadedApps->includes(self) implies
    (let CS:Snapshot = self.snapshotApplication in Set{CS}->closure(s | s.succ())->excluding(null)->exists(s | (let currentObject = s.application->any(o | o.id = self.id) in let previousObject = s.pred().application->any(o | o.id = self.id) in not (previousObject.system.installedApps->includes(previousObject)) and (currentObject.system.installedApps->includes(currentObject)))))
\end{lstlisting}


\subsection{Analysis of Results}

\hspace{1cm} Figure \ref{fig:object_diagram_case_study} shows a concrete scenario 
generated by the USE Model Validator when verifying our temporal properties. This 
scenario visualizes a complete application lifecycle through the software system, 
illustrating a sequence of operation calls: loading an application, installing it, 
running it, and finally stopping it. This execution path is particularly valuable 
as it exercises all four operations of our model in their expected sequence.

As shown in Figure \ref{fig:ocl_result_case_study}, all temporal properties are 
satisfied in this scenario. Each property verification confirms an important aspect 
of our system's behavior: safety1 verifies that the application was indeed loaded 
before being run; safety2 confirms the correct operational sequence was followed 
(load→install→run); safety3 validates that the application was loaded exactly once; 
and the liveness property confirms that after being loaded, the application was 
eventually installed.

These results demonstrate two important aspects of our approach. First, they validate 
the correctness of our transformation rules by confirming that the generated OCL 
constraints accurately encode the intended temporal semantics. Despite their 
complexity, the translated constraints correctly identify valid execution paths. 
Second, they show how our approach supports automated verification of complex 
temporal properties that would be impossible to express in standard OCL.

\begin{figure}[H]
    \centering
    \includegraphics[width=1\textwidth]{figures/c3/CaseStudy_ObjectDiagram_2.png}
    \caption{A scenario generated by the USE Model Validator.}
    \label{fig:object_diagram_case_study}
\end{figure}

\begin{figure}[H]
    \centering
    \includegraphics[width=0.4\textwidth]{figures/c3/OCL_result_full_1_edited.png}
    \caption{OCL result for generated scenario \ref{fig:object_diagram_case_study}.}
    \label{fig:ocl_result_case_study}
\end{figure}
\section{Discussion}

\hspace{1cm} Our TOCL+ approach for specifying and verifying temporal properties in 
UML/OCL models offers several advantages over existing approaches but also comes with 
certain limitations that warrant discussion.

The primary strength of our approach lies in its integration with standard modeling 
environments and tools. By extending OCL rather than introducing an entirely new 
formalism, we maintain compatibility with existing modeling practices while adding 
temporal verification capabilities. This integration allows modelers to work within 
familiar environments while gaining the ability to verify a broader class of 
properties. Our transformation-based approach also offers considerable flexibility. 
By converting TOCL+ specifications to standard OCL constraints on filmstrip models, 
we leverage established OCL tool capabilities without requiring specialized temporal 
verification engines. This provides a pragmatic solution that can be adopted without 
significant infrastructure changes or learning costs. The event-based constructs in 
TOCL+ address a significant gap in existing temporal OCL extensions. While previous 
approaches like TOCL provide temporal operators, they lack robust support for event 
detection and bounded existence properties. Our extensions make it possible to express 
important constraints related to operation calls and state changes, significantly 
broadening the range of verifiable properties.

While effective in practice, our approach has several technical limitations. The OCL 
constraints generated by our transformation can be complex, potentially affecting 
verification performance for large models with numerous temporal properties. These 
constraints involve intricate navigation through snapshots and careful handling of 
object identity, which may impact scalability for very large systems. Our approach 
also requires modelers to add identifier attributes to domain classes to maintain 
object identity across snapshots. This requirement introduces a small burden on 
modelers and slightly modifies the original domain model. A more elegant solution 
would be to handle object identity tracking automatically within the transformation 
framework. The current implementation also has limited support for complex expressions 
within event specifications. For example, nested event expressions or complex guards 
on events might not translate correctly in all cases. This restricts the sophistication 
of temporal properties that can be reliably verified.

From a methodological perspective, our work lacks a formal definition of the TOCL+ 
language and its transformation to OCL. Although our patterns appear to work correctly 
based on empirical evidence from case studies, we haven't provided formal proofs of 
semantic preservation between TOCL+ expressions and their OCL translations. This 
represents a theoretical gap that could affect confidence in the correctness of 
verification results for complex scenarios. Additionally, while our approach handles 
the core temporal operators and event constructs well, it doesn't yet support 
probabilistic or real-time properties. Systems with strict timing requirements or 
probabilistic behavior patterns would require extensions beyond the current 
capabilities.

Several promising research directions emerge from these limitations. The development 
of optimization techniques for the generated OCL constraints could significantly 
improve verification performance. Extending the approach with quantitative time 
constraints would enable verification of real-time properties. Creating a pattern 
library of common temporal verification scenarios with optimized translations would 
make the approach more accessible to practitioners. Enhancing the plugin with more 
intuitive visualizations of counterexamples when temporal properties are violated 
would improve usability. The formal definition of TOCL+ semantics and transformation 
rules, along with mathematical proofs of their correctness, would strengthen the 
theoretical foundations of the approach and provide stronger guarantees about 
verification results.
\section{Summary}
\hspace{1cm} In this chapter, we presented the practical implementation of our TOCL+ 
approach through a plugin for the USE tool and demonstrated its effectiveness through 
a detailed case study. The implementation bridges the gap between the theoretical 
foundations established in Chapter 2 and practical model verification by enabling 
modelers to specify temporal properties in a high-level language while leveraging 
existing OCL verification tools. Our plugin successfully implements the transformation 
rules that convert TOCL+ expressions into equivalent OCL constraints interpreted 
over filmstrip models. The implementation uses ANTLR4 for parsing TOCL+ expressions 
and employs a listener-based approach to systematically translate temporal operators 
and event constructs into structural OCL navigations.

The Software System case study demonstrated how our approach enables verification 
of diverse temporal properties that would be impractical to express in standard OCL. 
The safety properties successfully enforced correct operational sequencing and 
uniqueness constraints, while the liveness property verified eventual progress in 
the system. The complexity of the generated OCL constraints highlights the value of 
our approach: while these constraints involve intricate navigation through snapshots 
and require careful handling of object identity, the TOCL+ specification remains 
simple and intuitive.

Our discussion has identified both strengths and limitations of the approach. 
The primary strengths include seamless integration with standard modeling environments,
flexibility through the transformation to standard OCL, and enhanced expressiveness 
through event constructs. However, technical limitations such as complex generated 
constraints, object identity management requirements, and methodological limitations 
including the lack of formal definitions present opportunities for improvement. 
We have also outlined promising research directions including optimization techniques, 
support for real-time properties, and formal definition of the transformation semantics.

While our experiments provide empirical evidence supporting the correctness of the 
transformation rules, it's important to note that we have not formally proven 
the correctness of these translations mathematically. Nevertheless, the consistent 
verification results across different temporal properties and scenarios provide 
confidence in the practical applicability of the approach to realistic modeling 
scenarios. The verification performance remained acceptable despite the increased 
complexity of the generated constraints. This chapter thus validates our approach 
experimentally, showing how temporal verification can be effectively integrated 
into standard UML/OCL modeling environments without requiring specialized temporal 
verification tools.
% \section{Summary}
% \hspace{1cm} In this chapter, we presented the practical implementation of our TOCL+ 
% approach through a plugin for the USE tool and demonstrated its effectiveness through 
% a detailed case study. The implementation bridges the gap between the theoretical 
% foundations established in Chapter 2 and practical model verification by enabling 
% modelers to specify temporal properties in a high-level language while leveraging 
% existing OCL verification tools. Our plugin successfully implements the transformation 
% rules that convert TOCL+ expressions into equivalent OCL constraints interpreted 
% over filmstrip models. The implementation uses ANTLR4 for parsing TOCL+ expressions 
% and employs a listener-based approach to systematically translate temporal operators 
% and event constructs into structural OCL navigations.

% The Software System case study demonstrated how our approach enables verification 
% of diverse temporal properties that would be impractical to express in standard OCL. 
% The safety properties successfully enforced correct operational sequencing and 
% uniqueness constraints, while the liveness property verified eventual progress in 
% the system. The complexity of the generated OCL constraints highlights the value of 
% our approach: while these constraints involve intricate navigation through snapshots 
% and require careful handling of object identity, the TOCL+ specification remains 
% simple and intuitive. While our experiments provide empirical evidence supporting 
% the correctness of the transformation rules, it's important to note that we have not 
% formally proven the correctness of these translations mathematically. Nevertheless, 
% the consistent verification results across different temporal properties and scenarios 
% provide confidence in the practical applicability of the approach to realistic 
% modeling scenarios. The verification performance remained acceptable despite the 
% increased complexity of the generated constraints. This chapter thus validates our 
% approach experimentally, showing how temporal verification can be effectively integrated 
% into standard UML/OCL modeling environments without requiring specialized temporal 
% verification tools.
% % \hspace{1cm} In this chapter, we presented the practical implementation of our TOCL+ 
% % approach through a plugin for the USE tool and demonstrated its effectiveness through 
% % a detailed case study. The implementation bridges the gap between the theoretical 
% % foundations established in Chapter 2 and practical model verification by enabling 
% % modelers to specify temporal properties in a high-level language while leveraging 
% % existing OCL verification tools. Our plugin successfully implements the transformation 
% % rules that convert TOCL+ expressions into equivalent OCL constraints interpreted 
% % over filmstrip models. The implementation uses ANTLR4 for parsing TOCL+ expressions 
% % and employs a listener-based approach to systematically translate temporal operators 
% % and event constructs into structural OCL navigations.

% % The Software System case study demonstrated how our approach enables verification 
% % of diverse temporal properties that would be impractical to express in standard OCL. 
% % The safety properties successfully enforced correct operational sequencing and 
% % uniqueness constraints, while the liveness property verified eventual progress in 
% % the system. The complexity of the generated OCL constraints highlights the value of 
% % our approach: while these constraints involve intricate navigation through snapshots 
% % and require careful handling of object identity, the TOCL+ specification remains 
% % simple and intuitive. Our experiments confirm both the correctness of the 
% % transformation rules and the practical applicability of the approach to realistic 
% % modeling scenarios. The verification performance remained acceptable despite the 
% % increased complexity of the generated constraints. This chapter thus validates our 
% % approach not just theoretically but in practice, showing how temporal verification 
% % can be effectively integrated into standard UML/OCL modeling environments without 
% % requiring specialized temporal verification tools.

% 1. Introduction
% Brief overview of the chapter's objectives
% Connection to the theoretical foundation established in Chapter 2
% Outline of what follows in the chapter

% 2. Plugin Implementation
% Architecture and design of the TOCL+ plugin
% Implementation of the TOCL+ parser (generated from ANTLR4 grammar)
% Translation mechanisms for converting TOCL+ to OCL
% Technical details of the id attribute and object identity handling
% Integration with the USE environment

% 3. Software System Case Study
% Detailed description of the Software System model
% Specification of temporal properties using TOCL+
% Step-by-step verification walkthrough
% Results and analysis of verification outcomes
% Demonstration of how various temporal property types are verified

% 4. Summary
% Key findings from the implementation and experiment
% Assessment of the approach's effectiveness
% Implementation challenges and solutions\newpage\cleardoublepage
\begin{center}
  \textbf{\fontsize{20}{24}\selectfont Conclusion}
\end{center}

\addcontentsline{toc}{chapter}{Conclusion}

This thesis has addressed the challenge of specifying and verifying temporal 
properties in UML/OCL models without requiring specialized temporal verification 
tools. By extending OCL with temporal operators and event constructs, and providing 
a transformation mechanism to standard OCL, we have created an approach that enables 
modelers to verify complex behavioral properties while remaining within familiar 
modeling environments.

Our work has made several significant contributions to the field of model verification.
First, we introduced TOCL+ (Temporal OCL+), an extension of OCL that incorporates 
temporal operators and event constructs that enable expressing complex temporal 
properties in a concise and intuitive way. Second, we developed a transformation 
approach that converts TOCL+ expressions into equivalent standard OCL constraints 
that can be verified over filmstrip models, leveraging existing OCL tools without 
requiring specialized temporal logic model checkers.

Third, we implemented our approach as a plugin for the USE tool, demonstrating its 
practical applicability. The plugin uses ANTLR4 for parsing TOCL+ expressions and 
employs a listener-based approach to generate the corresponding OCL constraints.
Fourth, we validated our approach through a detailed case study of a Software System 
model, verifying various types of temporal properties including safety and liveness 
constraints.

The significance of our work lies in making temporal verification accessible within 
standard modeling environments. Our approach enables comprehensive model verification 
without requiring modelers to learn specialized temporal logics or verification tools.
This is particularly valuable for verifying behavioral properties that cannot be 
expressed in standard OCL, such as correct operational sequencing, eventual progress, 
and bounded existence constraints.

As discussed in Chapter 3, while our approach has proven effective in practice, 
several limitations exist and various directions for future research have been 
identified. These include developing formal proofs for our transformation rules, 
exploring optimization strategies, and extending support for real-time properties.

In conclusion, our TOCL+ approach successfully bridges the gap between standard 
UML/OCL modeling and temporal verification, enabling modelers to specify and verify 
complex behavioral properties without specialized temporal verification tools. 
The approach provides a practical solution to an important problem in model verification, 
with room for future improvements that can build upon the foundation established in 
this work.
% \begin{center}
%   \textbf{\fontsize{20}{24}\selectfont Conclusion}
% \end{center}

% \addcontentsline{toc}{chapter}{Conclusion}

% This thesis has addressed the challenge of specifying and verifying temporal 
% properties in UML/OCL models without requiring specialized temporal verification 
% tools. By extending OCL with temporal operators and event constructs, and providing 
% a transformation mechanism to standard OCL, we have created an approach that enables 
% modelers to verify complex behavioral properties while remaining within familiar 
% modeling environments.

% Our work has made several significant contributions to the field of model verification.
% First, we introduced TOCL+ (Temporal OCL+), an extension of OCL that incorporates 
% temporal operators (such as "always," "eventually," and "until") and event constructs 
% (like "becomesTrue" and "isCalled"). This extension enables modelers to express 
% complex temporal properties in a concise and intuitive way, addressing a significant 
% limitation of standard OCL. Second, we developed a transformation approach that converts 
% TOCL+ expressions into equivalent standard OCL constraints that can be verified over 
% filmstrip models. This approach leverages existing OCL tools for temporal verification 
% without requiring specialized temporal logic model checkers. The transformation 
% systematically maps temporal operators and event constructs to structural navigations 
% through snapshots, handling the complexities of object identity and state transitions.

% Third, we implemented our approach as a plugin for the USE tool, demonstrating its 
% practical applicability. The plugin uses ANTLR4 for parsing TOCL+ expressions and 
% employs a listener-based approach to generate the corresponding OCL constraints. 
% This implementation bridges the gap between the theoretical foundations and practical 
% verification tasks. Fourth, we validated our approach through a detailed case study 
% of a Software System model, verifying various types of temporal properties including 
% safety, uniqueness, and liveness constraints. The experimental results confirmed that 
% our transformation approach correctly preserves the semantics of temporal properties 
% while remaining practical for real-world modeling scenarios.

% The significance of our work lies in making temporal verification accessible within 
% standard modeling environments. By allowing modelers to specify temporal properties 
% at a high level of abstraction while handling the verification complexity behind 
% the scenes, we enable more comprehensive model verification without requiring 
% modelers to learn specialized temporal logics or verification tools. Our approach 
% is particularly valuable for verifying behavioral properties that cannot 
% be expressed in standard OCL, such as correct operational sequencing, eventual 
% progress, and bounded existence constraints. These properties are essential for 
% ensuring the correctness of dynamic system behavior, especially in domains where 
% operational sequences and timing constraints are critical.

% While our approach has proven effective in practice, several limitations should be 
% acknowledged. Most notably, we have not formally defined the transformation from TOCL+ 
% to OCL, nor have we provided a metamodel for the TOCL+ language extension. These would 
% provide a more rigorous foundation for our approach and enable formal analysis of the 
% language itself. Similarly, we have not formally proven the correctness of our 
% transformation rules mathematically. Although empirical evidence from our experiments 
% supports their correctness, a formal proof would provide stronger guarantees about 
% the preservation of temporal semantics. Additionally, the OCL constraints generated by 
% our transformation can be complex and potentially impact verification performance for 
% large models with numerous temporal properties. The constraints involve intricate 
% navigation through snapshots and careful handling of object identity, which may affect 
% scalability for very large systems.

% Several directions for future research emerge from our work. Developing formal proofs 
% for the correctness of our transformation rules would strengthen the theoretical 
% foundations of our approach. Exploring optimization strategies to reduce the complexity 
% of generated OCL constraints could improve verification performance for large models. 
% Creating a library of common temporal verification patterns with optimized translations 
% would make the approach more accessible to practitioners. Extending our approach to 
% other UML modeling tools beyond USE would broaden its applicability. Incorporating 
% quantitative time constraints would enable verification of real-time properties, 
% extending the approach to time-critical systems. Enhancing the plugin to provide 
% more intuitive visualizations of counterexamples when temporal properties are 
% violated would improve usability.

% In conclusion, our TOCL+ approach successfully bridges the gap between standard 
% UML/OCL modeling and temporal verification, enabling modelers to specify and verify 
% complex behavioral properties without specialized temporal verification tools. While 
% there remain opportunities for improvement and extension, the approach provides a 
% practical solution to an important problem in model verification.
\newpage\cleardoublepage

%\nocite{*}
% \phantomsection 
% \addcontentsline{toc}{chapter}{REFERENCES}
% \bibliography{references}\newpage\cleardoublepage
% \bibliographystyle{plain}

\printbibliography[title={REFERENCES}]
\addcontentsline{toc}{chapter}{REFERENCES}

\newpage
\begin{center}
\textbf{\fontsize{22}{0}\selectfont APPENDIX}
\end{center}
\addcontentsline{toc}{chapter}{APPENDIX}

\subsection*{A. The USE specification for Software System case study.}

\begin{lstlisting}[
    % caption={Specification of the Software System model in USE environment.},
    label={lst:software_system_specification}
]
model SoftwareSystem
-- Classes
class System
attributes
    id : Integer
    freeMemory : Integer init = 10
    loadedApps : Set(Application) init = Set{}
    installedApps : Set(Application) init = Set{}
    runningApps : Set(Application) init = Set{}
operations
    load(app : Application)
    begin
        self.loadedApps := self.loadedApps->including(app);
        self.freeMemory := self.freeMemory - app.size;
    end
    install()
    begin
        self.installedApps := self.installedApps->union(self.loadedApps);
        self.loadedApps := self.loadedApps->reject(true)->excluding(null);
    end
    run(app : Application)
    begin
        self.runningApps := self.runningApps->including(app);
    end
    stop(app : Application)
    begin
        self.runningApps := self.runningApps->excluding(app);
    end
end
class Application
attributes
    id : Integer
    size : Integer
end
-- Associations
association SystemApplication between
    System[1] role system
    Application[0..*] role apps
end
-- Invariants
constraints
context System
    inv memoryConstraint: self.freeMemory >= 0
    inv notLoadedAndInstalled: self.loadedApps->intersection(self.installedApps)->isEmpty()
    inv sets: let appNumber: Integer = self.apps->size() in
        (self.loadedApps->size() <= appNumber and
        self.installedApps->size() <= appNumber and
        self.runningApps->size() <= appNumber)
    inv notContainsNull:
        not self.loadedApps->includes(null) and
        not self.installedApps->includes(null) and
        not self.runningApps->includes(null)

context Application
    inv sizeConstraint: self.size > 0

context System::load(app: Application)
    pre notLoaded: not self.loadedApps->includes(app) and
                    not self.installedApps->includes(app) and
                    not self.runningApps->includes(app)
    pre enoughMemory: self.freeMemory >= app.size
    post loaded: self.loadedApps = self.loadedApps@pre->including(app)
    post freeMemory: self.freeMemory = self.freeMemory@pre - app.size
    post unchanged:
        self.apps->forAll(app |
            app.size = app.size@pre and
            app.id = app.id@pre)
    post sameInstalledAndRunning:
        self.installedApps = self.installedApps@pre and
        self.runningApps = self.runningApps@pre

context System::install()
    pre hasLoadedApps: self.loadedApps->notEmpty()
    post installed: self.installedApps = self.installedApps@pre->union(self.loadedApps@pre)
    post loadedAppsEmpty: self.loadedApps = self.loadedApps@pre->reject(true)->excluding(null)
    post sameRunning: self.runningApps = self.runningApps@pre
    post sameMemory: self.freeMemory = self.freeMemory@pre
    post unchanged:
        self.apps->forAll(app |
            app.size = app.size@pre and
            app.id = app.id@pre)

context System::run(app : Application)
    pre isInstalled: self.installedApps->includes(app)
    pre notRunning: not self.runningApps->includes(app)
    post running: self.runningApps = self.runningApps@pre->including(app)
    post sameLoaded: self.loadedApps = self.loadedApps@pre
    post sameInstalled: self.installedApps = self.installedApps@pre
    post sameMemory: self.freeMemory = self.freeMemory@pre
    post unchanged:
        self.apps->forAll(app |
            app.size = app.size@pre and
            app.id = app.id@pre)

context System::stop(app : Application)
    pre isRunning: self.runningApps->includes(app)
            and self.installedApps->includes(app)
    post notRunning: self.runningApps = self.runningApps@pre->excluding(app)
    post sameInstalled: self.installedApps = self.installedApps@pre
    post sameLoaded: self.loadedApps = self.loadedApps@pre
    post sameMemory: self.freeMemory = self.freeMemory@pre
    post unchanged:
        self.apps->forAll(app |
            app.size = app.size@pre and
            app.id = app.id@pre)
\end{lstlisting}

\newpage\cleardoublepage
\end{document}
