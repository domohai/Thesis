\begin{center}
  \textbf{\fontsize{20}{24}\selectfont Conclusion}
\end{center}

\addcontentsline{toc}{chapter}{Conclusion}

This thesis has addressed the challenge of specifying and verifying temporal 
properties in UML/OCL models without requiring specialized temporal verification 
tools. By extending OCL with temporal operators and event constructs, and providing 
a transformation mechanism to standard OCL, we have created an approach that enables 
modelers to verify complex behavioral properties while remaining within familiar 
modeling environments.

Our work has made several significant contributions to the field of model verification.
First, we introduced TOCL+ (Temporal OCL+), an extension of OCL that incorporates 
temporal operators and event constructs that enable expressing complex temporal 
properties in a concise and intuitive way. Second, we developed a transformation 
approach that converts TOCL+ expressions into equivalent standard OCL constraints 
that can be verified over filmstrip models, leveraging existing OCL tools without 
requiring specialized temporal logic model checkers.

Third, we implemented our approach as a plugin for the USE tool, demonstrating its 
practical applicability. The plugin uses ANTLR4 for parsing TOCL+ expressions and 
employs a listener-based approach to generate the corresponding OCL constraints.
Fourth, we validated our approach through a detailed case study of a Software System 
model, verifying various types of temporal properties including safety and liveness 
constraints.

The significance of our work lies in making temporal verification accessible within 
standard modeling environments. Our approach enables comprehensive model verification 
without requiring modelers to learn specialized temporal logics or verification tools.
This is particularly valuable for verifying behavioral properties that cannot be 
expressed in standard OCL, such as correct operational sequencing, eventual progress, 
and bounded existence constraints.

As discussed in Chapter 3, while our approach has proven effective in practice, 
several limitations exist and various directions for future research have been 
identified. These include developing formal proofs for our transformation rules, 
exploring optimization strategies, and extending support for real-time properties.

In conclusion, our TOCL+ approach successfully bridges the gap between standard 
UML/OCL modeling and temporal verification, enabling modelers to specify and verify 
complex behavioral properties without specialized temporal verification tools. 
The approach provides a practical solution to an important problem in model verification, 
with room for future improvements that can build upon the foundation established in 
this work.
% \begin{center}
%   \textbf{\fontsize{20}{24}\selectfont Conclusion}
% \end{center}

% \addcontentsline{toc}{chapter}{Conclusion}

% This thesis has addressed the challenge of specifying and verifying temporal 
% properties in UML/OCL models without requiring specialized temporal verification 
% tools. By extending OCL with temporal operators and event constructs, and providing 
% a transformation mechanism to standard OCL, we have created an approach that enables 
% modelers to verify complex behavioral properties while remaining within familiar 
% modeling environments.

% Our work has made several significant contributions to the field of model verification.
% First, we introduced TOCL+ (Temporal OCL+), an extension of OCL that incorporates 
% temporal operators (such as "always," "eventually," and "until") and event constructs 
% (like "becomesTrue" and "isCalled"). This extension enables modelers to express 
% complex temporal properties in a concise and intuitive way, addressing a significant 
% limitation of standard OCL. Second, we developed a transformation approach that converts 
% TOCL+ expressions into equivalent standard OCL constraints that can be verified over 
% filmstrip models. This approach leverages existing OCL tools for temporal verification 
% without requiring specialized temporal logic model checkers. The transformation 
% systematically maps temporal operators and event constructs to structural navigations 
% through snapshots, handling the complexities of object identity and state transitions.

% Third, we implemented our approach as a plugin for the USE tool, demonstrating its 
% practical applicability. The plugin uses ANTLR4 for parsing TOCL+ expressions and 
% employs a listener-based approach to generate the corresponding OCL constraints. 
% This implementation bridges the gap between the theoretical foundations and practical 
% verification tasks. Fourth, we validated our approach through a detailed case study 
% of a Software System model, verifying various types of temporal properties including 
% safety, uniqueness, and liveness constraints. The experimental results confirmed that 
% our transformation approach correctly preserves the semantics of temporal properties 
% while remaining practical for real-world modeling scenarios.

% The significance of our work lies in making temporal verification accessible within 
% standard modeling environments. By allowing modelers to specify temporal properties 
% at a high level of abstraction while handling the verification complexity behind 
% the scenes, we enable more comprehensive model verification without requiring 
% modelers to learn specialized temporal logics or verification tools. Our approach 
% is particularly valuable for verifying behavioral properties that cannot 
% be expressed in standard OCL, such as correct operational sequencing, eventual 
% progress, and bounded existence constraints. These properties are essential for 
% ensuring the correctness of dynamic system behavior, especially in domains where 
% operational sequences and timing constraints are critical.

% While our approach has proven effective in practice, several limitations should be 
% acknowledged. Most notably, we have not formally defined the transformation from TOCL+ 
% to OCL, nor have we provided a metamodel for the TOCL+ language extension. These would 
% provide a more rigorous foundation for our approach and enable formal analysis of the 
% language itself. Similarly, we have not formally proven the correctness of our 
% transformation rules mathematically. Although empirical evidence from our experiments 
% supports their correctness, a formal proof would provide stronger guarantees about 
% the preservation of temporal semantics. Additionally, the OCL constraints generated by 
% our transformation can be complex and potentially impact verification performance for 
% large models with numerous temporal properties. The constraints involve intricate 
% navigation through snapshots and careful handling of object identity, which may affect 
% scalability for very large systems.

% Several directions for future research emerge from our work. Developing formal proofs 
% for the correctness of our transformation rules would strengthen the theoretical 
% foundations of our approach. Exploring optimization strategies to reduce the complexity 
% of generated OCL constraints could improve verification performance for large models. 
% Creating a library of common temporal verification patterns with optimized translations 
% would make the approach more accessible to practitioners. Extending our approach to 
% other UML modeling tools beyond USE would broaden its applicability. Incorporating 
% quantitative time constraints would enable verification of real-time properties, 
% extending the approach to time-critical systems. Enhancing the plugin to provide 
% more intuitive visualizations of counterexamples when temporal properties are 
% violated would improve usability.

% In conclusion, our TOCL+ approach successfully bridges the gap between standard 
% UML/OCL modeling and temporal verification, enabling modelers to specify and verify 
% complex behavioral properties without specialized temporal verification tools. While 
% there remain opportunities for improvement and extension, the approach provides a 
% practical solution to an important problem in model verification.
