\begin{center}
  \textbf{\fontsize{20}{24}\selectfont KẾT LUẬN}
\end{center}

\addcontentsline{toc}{chapter}{KẾT LUẬN}

Khóa luận này đặt trọng tâm vào việc phát triển một nền tảng tạo lập và giao
dịch token trên các chuỗi khối khác nhau. Kết quả đạt
được của khóa luận đã giúp phát triển một sàn giao dịch phi tập trung với nhiều
tính năng bổ sung độc đáo hỗ trợ mạnh mẽ hệ sinh thái tài chính phi tập trung
như tạo mới, giao dịch hay gửi tiết kiệm các loại token khác nhau. Những tính
năng này được nghiên cứu
từ hiện trạng và nhu cầu thực tế và đáp ứng được các mục tiêu đã đặt ra bao
gồm:
Tối ưu hóa chi phí tạo token, tăng cường độ tin cậy của các dịch vụ tài chính
thông qua việc sử dụng hợp đồng thôn minh.
Kết quả nghiên cứu, phát triển và kiểm thử cho thấy rằng hệ thống hoạt động ổn
định và đáp ứng các yêu cầu chức năng và phi chức năng ban đầu đã đặt ra. Hệ
thống cũng có giao diện thân thiện và nhận được phản hồi tích cực từ người dùng
mẫu.
Mặc dù đã đạt được các mục tiêu đề ra, nền tảng LaunchCrypt vẫn phải đối mặt
với một số thách thức. Đầu tiên là tốc độ xử lý có thể chưa đạt
được mức độ nhanh nhất để đáp ứng nhu cầu của người dùng thực tế. Điều này
gây ra bởi môi trường, tài nguyên và thời gian phát triển còn hạn chế. Hơn nữa,
quá trình vận hành và kiểm thử diễn ra chủ yếu trong môi trường phát triển và
cộng đồng người dùng thử nghiệm còn nhỏ. Do đó, hệ thống có thể không đáp ứng
hoàn toàn nhu cầu sử dụng thực thế và cần thiết được nghiên cứu và phát triển
thêm. Khi các hạn chế và tài nguyên thiết bị được cải thiện, hệ thống có thể
đáp
ứng được nhu cầu sử dụng của người dùng thực tế.

\hspace{-1cm}\textbf{Phương hướng phát triển trong tương lai}

Trong quá trình tiến hành nghiên cứu và phát triển nền tảng LaunchCrypt,
khóa luận đã xem xét các phương hướng phát triển trong tương lai để nâng cao
hiệu quả sử dụng và tiếp cận thêm được nhiều người dùng cuối. Các phương hướng
này không chỉ là cơ hội để cải thiện hiệu suất hoạt động của hệ thống mà còn là
cơ hội để mở rộng phạm vi ứng dụng và tạo ra những trải nghiệm người dùng tốt
hơn. Trước hết, hệ thống sẽ hoàn thành việc triển khai trên các chuỗi khối
layer 1 khác nhau, bao gồm Aptos và Solana. Điều này sẽ giúp tăng lượng người
dùng mới, giúp hệ sinh thái của dự án trở nên đa dạng và phổ biến hơn.

\hspace{-1cm}Tiếp theo, hệ thống có thể tích hợp các tính năng liên quan đến
thanh toán trực tiếp bằng tiền pháp định (fiat). Điều này sẽ giúp nền tảng mở
rộng không chỉ dừng lại ở tài chính phi tập trung mà còn hướng tới mảng tài
chính truyền thống.

\hspace{-1cm}Cuối cùng, sử dụng AI để giúp người dùng thống kê thông số của các
token hay lên các chiến lược giao dịch là một mục tiêu mà nền tảng hướng tới.
Điều này không chỉ giúp những người dùng mới sử dụng dịch vụ tốt hơn mà còn
giúp những người dùng đã có kinh nghiệm dễ dàng phân tích và có một cái nhìn
tổng quan về hoạt động của nền tảng cũng như dữ liệu về những loại token mà họ
quan tâm.
