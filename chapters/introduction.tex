\setcounter{page}{1}

\pagenumbering{arabic}

\setlength{\parindent}{1cm}

\begin{center}
  \section*{INTRODUCTION}
\end{center}
\addcontentsline{toc}{chapter}{INTRODUCTION}

Modern software development faces significant challenges as
systems grow increasingly complex. Traditional development approaches relying on 
manual coding often struggle to manage this complexity, leading to higher error
rates and extended development cycles. These problems often come from the development 
process, not the system requirements. Model-Driven Engineering (MDE) \cite{MDE} 
addresses these challenges by shifting the focus from code to models. 
In MDE, developers use models to design systems, and tools can automatically generate 
code, documentation, and tests from those models. 
The Unified Modeling Language (UML) \cite{UML} and the Object 
Constraint Language (OCL) \cite{OCL} have become the \textit{de facto} standards 
for model-driven approaches. UML provides a rich set of visual modeling concepts to 
represent the structural and behavioral aspects of a system, while OCL allows 
specifying constraints and structural properties of UML models. However, for 
complex systems, it is often necessary to specify and verify dynamic behaviors 
that involve temporal constraints and event-driven conditions. Unfortunately, OCL 
lacks the expressiveness to model these dynamic aspects, which limits its ability 
to specify and verify temporal properties and event-based behaviors.

This thesis aims to address this limitation by extending OCL with constructs for
temporal properties and events, thereby enhancing its expressiveness in modeling dynamic 
system aspects. We implement this extension, called TOCL+, as a plugin for the 
UML-based Specification Environment (USE) \cite{USE}, a tool that supports the 
specification and validation of software systems using UML and OCL. To enable both
specification and verification of temporal properties, we employ a 
technique known as filmstripping \cite{Filmstripping}, which transforms models 
with dynamic temporal constraints into structurally equivalent models that can 
be analyzed using existing verification tools. Our plugin automatically translates 
temporal OCL expressions into standard OCL constraints on filmstrip models, 
allowing modelers to leverage the existing USE model validator \cite{USE_Validator} 
for verification. This approach bridges the gap between expressing temporal 
requirements and verifying them, providing a complete solution that integrates 
seamlessly with the established USE environment and its validation capabilities.
% This thesis aims to address this limitation by extending OCL with constructs for
% temporal properties and events, enhancing its expressiveness in modeling dynamic 
% system aspects. We implement this extension as a plugin, called TOCL+, for the 
% UML-based Specification Environment (USE) \cite{USE}, a tool that supports the 
% specification and validation of software systems using UML and OCL. To enable not 
% only specification but also verification of temporal properties, we employ a 
% technique known as filmstripping \cite{Filmstripping}, which transforms models 
% with dynamic temporal constraints into structurally equivalent models that can 
% be analyzed using existing verification tools. Our plugin automatically translates 
% temporal OCL expressions into standard OCL constraints on a filmstrip model, 
% allowing modelers to leverage the existing USE model validator \cite{USE_Validator} 
% for verification. This approach bridges the gap between expressing temporal 
% requirements and verifying them, providing a complete solution that integrates 
% seamlessly with the established USE environment and its validation capabilities.

The thesis is structured as follows:
\begin{itemize}
  \item \textbf{Chapter 1: Background} lays the foundation for understanding the theoretical concepts and tools used in this thesis.
  
  \item \textbf{Chapter 2: Specification and Verification of Temporal Properties in OCL} presents our approach to extending OCL for specifying and verifying temporal properties.

  \item \textbf{Chapter 3: Implementation and Experiment} describes the implementation of our approach and evaluates it through a case study.

  \item \textbf{Conclusion} summarizes the contributions of this thesis and discusses future work.
\end{itemize}