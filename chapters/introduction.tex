\setcounter{page}{1}

\pagenumbering{arabic}

\setlength{\parindent}{1cm}

\begin{center}
  \section*{MỞ ĐẦU}
\end{center}

\subsection*{Bối cảnh và tầm quan trọng của đề tài}

\hspace{\parindent}Trong những năm gần đây, sự phát triển mạnh mẽ của công nghệ
chuỗi khối đã mở ra một kỷ nguyên mới cho lĩnh vực tài chính, đặc biệt là với
sự xuất hiện của tài chính Phi tập trung (DeFi). DeFi đã và đang thay đổi cách
chúng ta nghĩ về tiền tệ, giao dịch và đầu tư, mang lại cơ hội cho hàng triệu
người trên toàn cầu tiếp cận với các dịch vụ tài chính một cách công bằng và
minh bạch hơn.

Tuy nhiên, bất chấp tiềm năng to lớn, DeFi vẫn phải đối mặt với nhiều thách
thức đáng kể. Một trong những rào cản lớn nhất là sự phức tạp về mặt kỹ thuật,
khiến nhiều người dùng, đặc biệt là những người không có nền tảng công nghệ,
gặp khó khăn trong việc tham gia vào hệ sinh thái này. Bên cạnh đó, các vấn đề
như rug pull, khó tiếp cận với thị trường và các hình thức lừa đảo khác đang
làm giảm lòng tin của người dùng và cản trở sự phát triển rộng rãi của DeFi.

Trong bối cảnh đó, việc phát triển một nền tảng như LaunchCrypt - một hệ sinh
thái toàn diện cho phép người dùng dễ dàng tạo, giao dịch và phát triển token
của riêng họ một cách an toàn - trở nên cấp thiết và có ý nghĩa to lớn.
LaunchCrypt không chỉ đơn thuần là một công cụ kỹ thuật, mà còn là một bước
tiến quan trọng trong việc hỗ trợ những người muốn phát triển token của riêng
mình cũng như mở rộng khả năng tiếp cận DeFi cho đông đảo người dùng khác.

\subsection*{Mục tiêu nghiên cứu}

\hspace{\parindent}Nghiên cứu này hướng tới việc phát triển một giải pháp toàn
diện nhằm đẩy mạnh việc áp dụng rộng rãi công nghệ chuỗi khối và tài chính phi tập trung (DeFi).
Mục tiêu chính là phân tích và đánh giá các thách thức hiện tại trong việc tạo và giao
dịch token trên các nền tảng DeFi, từ đó thiết kế và phát triển một nền tảng
cho phép người dùng không có kiến thức chuyên sâu về công nghệ có thể dễ dàng
tham gia vào hệ sinh thái này. Nền tảng này không chỉ tập trung vào việc đơn
giản hóa quá trình tạo và giao dịch token, mà còn chú trọng xây dựng các cơ chế
bảo mật và quản lý rủi ro hiệu quả, nhằm đảm bảo an toàn cho người dùng và ngăn
chặn các hoạt động gian lận. Đồng thời, nghiên cứu cũng hướng tới việc tích hợp
khả năng hỗ trợ đa chuỗi, bao gồm Aptos, Solana và Avalanche, nhằm tăng tính
linh hoạt và khả năng tương tác của nền tảng. Một mục tiêu quan trọng khác là
tích hợp hỗ trợ cho người dùng web 2.0, góp phần thu hẹp khoảng cách giữa tài
chính truyền thống và tài chính phi tập trung. Cuối cùng, nghiên cứu sẽ đánh
giá hiệu quả của ứng dụng trong việc giải quyết các thách thức hiện tại của
DeFi và đề xuất hướng phát triển tiếp theo, nhằm đóng góp vào sự phát triển bền
vững của lĩnh vực này trong tương lai.

\subsection*{Phương pháp nghiên cứu}
Để đạt được các mục tiêu trên, khóa luận sẽ áp dụng các phương pháp nghiên cứu
sau:
\begin{itemize}
  \item \textbf{Nghiên cứu lý thuyết}: Tổng hợp và phân tích các tài liệu,
        nghiên cứu liên quan đến công nghệ chuỗi khối, tài chính phi tập trung và phát triển token.

  \item \textbf{Phát triển phần mềm}: Sử dụng các công nghệ như ReactJS,
        NestJS, và các ngôn ngữ lập trình hợp đồng thông minh (Move, Solidity, Rust) để xây
        dựng nền tảng LaunchCrypt.

  \item \textbf{Thử nghiệm và đánh giá}: Triển khai và thử nghiệm nền tảng
        trong môi trường thử nghiệm, thu thập và phân tích dữ liệu về hiệu suất và trải
        nghiệm người dùng.
\end{itemize}

\subsection*{Cấu trúc của khóa luận}
Cấu trúc khóa luận gồm 4 phần như sau:
\begin{itemize}
  \item \textbf{Chương 1}: Trình bày các kiến thức nền tảng về công nghệ chuỗi khối và
        tài chính phi tập trung (DeFi). Giới thiệu sơ lược về các công nghệ hỗ trợ sử
        dụng trong quá trình xây dựng sản phẩm.

  \item \textbf{Chương 2}: Đưa ra các thiết kế về phần mềm, các thành phần và
        mô tả tương tác giữa chúng.

  \item \textbf{Chương 3}: Cài đặt và thực nghiệm, áp dụng và triển khai phần
        mềm, vận dụng và trình bày kết quả sản phẩm.

  \item \textbf{Kết luận}: Kết quả đạt được và hướng phát triển; Đưa ra tổng
        kết về đề tài, các đề xuất và hướng nghiên cứu tiếp theo.
\end{itemize}