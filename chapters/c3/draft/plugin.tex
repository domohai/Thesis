\subsection{Plugin Architecture and Workflow}
% \hspace{1cm} Figure \ref{sec:plugin_architecture} illustrates the architecture 
% and workflow of our TOCL+ plugin. The plugin is designed to seamlessly 
% integrate with the USE tool environment while maintaining a clear separation 
% between the application model, the temporal property specification, and the 
% verification process. The workflow begins with the user preparing two input files: 
% a standard \texttt{.use} file containing the UML/OCL application model and a 
% \texttt{.tocl} file specifying the TOCL+ properties to be verified. After loading 
% the application model into USE, the user activates our plugin through the USE 
% interface and is prompted to select both a destination file for the output model 
% and the \texttt{.tocl} file containing the temporal property definitions. Internally, 
% the plugin orchestrates a two-phase transformation process. First, it invokes the 
% existing Filmstrip plugin to transform the application model into a filmstrip model 
% according to the rules described in Section~\ref{subsec:filmstripping}. Then, it 
% processes the TOCL+ expressions using our ANTLR4-generated parser and corresponding 
% listeners that implement the transformation rules for converting temporal 
% specifications into equivalent OCL constraints, which will be detailed in the next 
% subsection. These generated constraints are added to the list of constraints in the output model 
% file alongside the filmstrip model. To complete the verification process, the user loads 
% this output model back into USE together with the configuration file to employ the Model Validator to analyze the constraints. 
% This architecture ensures a clean separation of concerns while providing an integrated 
% workflow that shields users from the complexities of the underlying transformation mechanisms.