\subsection{Plugin Architecture and Workflow}
%%% Version 1
% \hspace{1cm} Figure \ref{sec:plugin_architecture} illustrates the architecture 
% and workflow of our TOCL+ plugin. The plugin is designed to seamlessly 
% integrate with the USE tool environment while maintaining a clear separation 
% between the application model, the temporal property specification, and the 
% verification process. The workflow begins with the user preparing two input files: 
% a standard \texttt{.use} file containing the UML/OCL application model and a 
% \texttt{.tocl} file specifying the TOCL+ properties to be verified. After loading 
% the application model into USE, the user activates our plugin through the USE 
% interface and is prompted to select both a destination file for the output model 
% and the \texttt{.tocl} file containing the temporal property definitions. Internally, 
% the plugin orchestrates a two-phase transformation process. First, it invokes the 
% existing Filmstrip plugin to transform the application model into a filmstrip model 
% according to the rules described in Section~\ref{subsec:filmstripping}. Then, it 
% processes the TOCL+ expressions using our ANTLR4-generated parser and corresponding 
% listeners that implement the transformation rules for converting temporal 
% specifications into equivalent OCL constraints, which will be detailed in the next 
% subsection. These generated constraints are added to the list of constraints in the output model 
% file alongside the filmstrip model. To complete the verification process, the user loads 
% this output model back into USE together with the configuration file to employ the Model Validator to analyze the constraints. 
% This architecture ensures a clean separation of concerns while providing an integrated 
% workflow that shields users from the complexities of the underlying transformation mechanisms.

%%% Version 2
% \hspace{1cm} Figure \ref{sec:plugin_support_tool_architecture} illustrates the architecture and 
% workflow of our TOCL+ support tool. The support tool integrates with the USE tool environment 
% while maintaining a clear separation between modeling, specification, and 
% verification concerns. The workflow consists of five distinct steps, beginning with 
% preparation and ending with verification. First, the user prepares two input files: 
% a standard \texttt{.use} file containing the UML/OCL application model and a 
% \texttt{.tocl} file containing TOCL+ property specifications. Second, the user loads 
% the application model into USE to make it available for transformation. Third, 
% the user activates our plugin through the USE interface, selecting both a destination 
% path for the output model and the \texttt{.tocl} file containing the temporal 
% properties to verify.

% Internally, the plugin then executes a two-phase transformation process. In the first 
% phase, it invokes the Filmstrip plugin to transform the application model into a 
% filmstrip model following the rules described in Section~\ref{subsec:filmstripping}. 
% In the second phase, it processes the TOCL+ expressions using our ANTLR4-generated 
% parser and listener components, which implement the transformation rules for 
% converting temporal specifications into equivalent OCL constraints (detailed in the 
% next subsection). These generated constraints are added to the list of invariants in 
% the output model file alongside the filmstrip model elements.

% To complete the verification process, the user loads this output model back into USE 
% together with a configuration file that establishes search bounds, and then employs 
% the Model Validator to analyze the constraints. The validator systematically explores the search space, determining whether the temporal properties are satisfied and providing a model instance as evidence when applicable. This architecture shields users from the complexities of the underlying transformation mechanisms while providing a streamlined workflow from specification to verification.
