\subsection{Plugin Architecture and Workflow}
%%% Version 1
% \hspace{1cm} Figure \ref{sec:plugin_architecture} illustrates the architecture 
% and workflow of our TOCL+ plugin. The plugin is designed to seamlessly 
% integrate with the USE tool environment while maintaining a clear separation 
% between the application model, the temporal property specification, and the 
% verification process. The workflow begins with the user preparing two input files: 
% a standard \texttt{.use} file containing the UML/OCL application model and a 
% \texttt{.tocl} file specifying the TOCL+ properties to be verified. After loading 
% the application model into USE, the user activates our plugin through the USE 
% interface and is prompted to select both a destination file for the output model 
% and the \texttt{.tocl} file containing the temporal property definitions. Internally, 
% the plugin orchestrates a two-phase transformation process. First, it invokes the 
% existing Filmstrip plugin to transform the application model into a filmstrip model 
% according to the rules described in Section~\ref{subsec:filmstripping}. Then, it 
% processes the TOCL+ expressions using our ANTLR4-generated parser and corresponding 
% listeners that implement the transformation rules for converting temporal 
% specifications into equivalent OCL constraints, which will be detailed in the next 
% subsection. These generated constraints are added to the list of constraints in the output model 
% file alongside the filmstrip model. To complete the verification process, the user loads 
% this output model back into USE together with the configuration file to employ the Model Validator to analyze the constraints. 
% This architecture ensures a clean separation of concerns while providing an integrated 
% workflow that shields users from the complexities of the underlying transformation mechanisms.

%%% Version 2
% \hspace{1cm} Figure \ref{sec:plugin_support_tool_architecture} illustrates the architecture and 
% workflow of our TOCL+ support tool. The support tool integrates with the USE tool environment 
% while maintaining a clear separation between modeling, specification, and 
% verification concerns. The workflow consists of five distinct steps, beginning with 
% preparation and ending with verification. First, the user prepares two input files: 
% a standard \texttt{.use} file containing the UML/OCL application model and a 
% \texttt{.tocl} file containing TOCL+ property specifications. Second, the user loads 
% the application model into USE to make it available for transformation. Third, 
% the user activates our plugin through the USE interface, selecting both a destination 
% path for the output model and the \texttt{.tocl} file containing the temporal 
% properties to verify.

% Internally, the plugin then executes a two-phase transformation process. In the first 
% phase, it invokes the Filmstrip plugin to transform the application model into a 
% filmstrip model following the rules described in Section~\ref{subsec:filmstripping}. 
% In the second phase, it processes the TOCL+ expressions using our ANTLR4-generated 
% parser and listener components, which implement the transformation rules for 
% converting temporal specifications into equivalent OCL constraints (detailed in the 
% next subsection). These generated constraints are added to the list of invariants in 
% the output model file alongside the filmstrip model elements.

% To complete the verification process, the user loads this output model back into USE 
% together with a configuration file that establishes search bounds, and then employs 
% the Model Validator to analyze the constraints. The validator systematically explores the search space, determining whether the temporal properties are satisfied and providing a model instance as evidence when applicable. This architecture shields users from the complexities of the underlying transformation mechanisms while providing a streamlined workflow from specification to verification.

%%% Version 3
% For the transformation to work correctly, we needed to establish fundamental 
% navigation operations that allow constraints to traverse the filmstrip model. 
% The key operations include: \texttt{self.snapshot} to get the snapshot associated 
% with an object in the current state; \texttt{snapshot.pred()} and 
% \texttt{snapshot.succ()} to navigate to the previous and next snapshots; and 
% \texttt{object.pred} and \texttt{object.succ} to navigate to the corresponding 
% object in the previous and next states.

% A significant challenge in the filmstrip approach is maintaining object identity 
% across different snapshots. While the filmstrip model provides \texttt{pred} and 
% \texttt{succ} associations to navigate between objects in adjacent snapshots, it 
% doesn't inherently provide a mechanism to identify the same conceptual object across 
% non-adjacent states. To address this limitation, we require modelers to add an 
% \texttt{id} attribute to all classes in the application model. As seen in Figure 
% \ref{fig:object_diagram_liveness}, this allows us to identify the same logical 
% object (e.g., three Application objects with id 0) across different snapshots. 
% Our plugin adds additional constraints to ensure this id remains consistent 
% throughout the timeline, enabling reliable temporal navigation.

% Table \ref{tab:TOCL2OCL} presents the complete set of translation patterns we 
% developed for TOCL+ operators and event constructs. Each pattern systematically 
% maps a TOCL+ construct to an equivalent OCL expression interpreted over the filmstrip 
% model. The patterns use placeholders (indicated by square brackets) that get 
% substituted during the transformation process: \texttt{[s |= P]} indicates that 
% property P holds in snapshot s; \texttt{[ContextSnapshot]} is the snapshot in which 
% the property is evaluated; \texttt{[ContextObject]} is the object on which the 
% property is evaluated; and \texttt{[OpClassName]} is the class representing an 
% operation call in the filmstrip model. The bounded quantifiers (e.g., \texttt{at most},
% \texttt{at least}) are translated into corresponding comparators. If there are no
% quantifiers specified, we implicitly assume that the event happens exactly n times.


%%% Sample paragraph 1
% Using ANTLR4 listeners, we implemented the transformation as Java code built on 
% the TOCL parser we created. The listeners traverse the parse tree of a TOCL+
% expression and produce the corresponding OCL expression. The transformation from TOCL to OCL is implemented by overriding generated listener methods of the
% ANTLR4 TOCLparser. To use these rules we walk the parse tree generated by the TOCL parser.
% The above implementation is inspired by \cite{TOCL2OCL}, in their work, they 
% transformed TOCL \cite{TOCL} into OCL in the context of a STM which is another kind
% of class model to represent dynamic properties in the form of static properties. We 
% adopted their approach and applied it in our approach which transform TOCL+ to OCL 
% within the context of filmtrip model.
%%% Sample paragraph 2
% To transform TOCL+ expressions, we defined translations for TOCL operators and events
% to OCL, as shown in Table \ref{tab:TOCL2OCL}. To create these translations, 
% we utilized some query operations. The self.snapshot query gets the snapshot
% associated with an object in the "current state" or the state where the expression 
% is being evaluated. The pred() and succ() operations when applied to a snapshot get 
% the snapshot in the previous and next state, respectively. For objects to navigate
% to the corresponding object in the next and previous state, we use the two associations
% .succ and .pred. Note that the parts of the patterns that are between square brackets
% (e.g. [s |= P], [ContextSnapshot], [ContextObject]) must be replaced when instances 
% of the patterns are specified. The expression [s |= P] in the OCL patterns mean that 
% the property P holds in the snapshot s. P is a boolean expression in the context of 
% a class, for example C. When using the patterns to write temporal properties, the OCL 
% expression representing s |= P is generated by navigating from the snapshot s to C,
% the context of OCL expression P. The [ContextSnapshot] and [ContextObject] will be 
% replaced with the corresponding snapshot and object when the model is transformed. 
% For example, in the translation of the liveness property shown in Listing
% \ref{lst:ocl_liveness} the [ContextSnapshot] is replaced with the local vairable s
% inside the exists query. 
% Note that the use id attribute. Recall from Fig. \ref{fig:object_diagram_liveness},
% there are three Application objects with id 0 and three with id 1. Because filmstrip 
% model does not provide any means to identify the same object across different states, 
% it only provides the pred and succ associations to navigate between them. In order to 
% overcome the above drawback of filmstrip model, the modeler must add id attribute to all the 
% application model classes. And internally when TOCL+ plugin transforms the model
% we add additional constraints to make sure the id stays consistent between different 
% states. And the id attribute is put into use in the OCL translation for the becomesTrue
% event. The result of navigation self.snapshot.[ContextObject] is a collection of
% objects of the same type, we only want to select the object with the same id as the
% object we are evaluating the property on so we use ->any(o | o.id = self.id). 
%%% Sample paragraph 3
% After defining these translations, we created rules for them using listeners. These 
% rules take the OCL translation of each TOCL operator and simply replace the 
% appropriate parts with the children of the corresponding parse tree node.
% For instance, when translating a becomesTrue event, we extract the boolean expression
% and construct the corresponding OCL expression to be evaluated. We do this using
% getOCL(ParseTree ctx) operation, which gets the text associated with node ctx. In 
% addition, we also keep track of the original TOCL+ expression in the variable
% originalEvent. After the translation, we push both the translated event expression
% and the original event expression into a stack, which will be popped at the end of 
% traversing the entire parse tree. Listing \ref{lst:becomesTrue_translation} shows 
% an example of the implementation of the becomesTrue event using an ANTLR4 listener.

%%% Sample paragraph 4
% Once the listener has visited every node within a tree, it finally
% visits the root where it creates the string that is the result of the
% OCL translation. Since TOCL is an extension of OCL, many of
% the constructs are the same and should be directly mapped in a
% translation. Thus, first, we store the token stream in the variable
% tokens. The operation getText(ctx) concatenates the lexemes of
% the tokens that descend from the node represented by ctx. Next,
% we pop every translated OCL expression and the corresponding
% original TOCL expression from the aforementioned stack. Then,
% we replace every instance of the original TOCL expression with its
% equivalent OCL translation using the replace(CharSequence target,
% CharSequence replacement) operation. Finally, after replacing the
% translated TOCL expressions, we use the setOCL(ParseTree ctx,
% String s) operation to set this finished translated OCL to the root of
% the parse tree

% \begin{table}[htbp]
%     \caption{Translation of TOCL+ operators and events to OCL.}
%     \label{tab:TOCL2OCL}
%     \begin{tabularx}{\textwidth}{|>{\footnotesize}p{0.6cm}|>{\scriptsize\raggedright\arraybackslash}p{4cm}|>{\scriptsize\raggedright\arraybackslash}X|}
%         \hline
%         \textbf{No.} & \textbf{TOCL+} & \textbf{OCL Translation} \\
%         \hline
%         1 & 
%         next P &
%         let nextSnapshot:Snapshot = self.snapshot.succ() in [nextSnapshot |= P] \\
%         \hline
%         2 &
%         always P &
%         let CS:Snapshot = self.snapshot in Set\{CS\}->closure(s | s.succ())->forAll(s | [s |= P]) \\
%         \hline  
%         3 &
%         always P until Q &
%         let CS:Snapshot = self.snapshot
%         in let FS:Set(Snapshot) = Set\{CS.succ()\}->closure(s | s.succ())
%         in let AllFSQ:Set(Snapshot) = FS->select(s | [s |= Q])
%         in let FSQ:Snapshot = AllFSQ->any(s | Set\{s\}->closure(s | s.succ())->includesAll(AllFSQ))
%         in let afterQ:Set(Snapshot) = Set\{FSQ\}->closure(s | s.succ())
%         in let FSP:Set(Snapshot) = FS->select(s | [s |= P])
%         in if FSQ.isDefined() then (if (FSP->size() > 0) then (FS-afterQ = FSP-afterQ) else false endif) else (FS = FSP) endif \\
%         \hline
%         4 &
%         always P since Q &
%         let CS:Snapshot = self.snapshot
%         in let PS:Set(Snapshot) = Set\{CS.pred()\}->closure(s | s.pred())
%         in let AllPSQ:Set(Snapshot) = PS->select(s | [s |= Q])
%         in let PSQ:Snapshot = AllPSQ->any(s | Set\{s\}->closure(s | s.pred())->includesAll(AllPSQ))
%         in let beforeQ:Set(Snapshot) = Set\{PSQ\}->closure(s | s.pred())
%         in let PSP:Set(Snapshot) = PS->including(CS)->select(s | [s |= P])
%         in if PSQ.isDefined() then (if (PSP->size() > 0) then (PS->including(CS)-beforeQ = PSP-beforeQ) else false endif) else (PSP = PS->including(CS)) endif \\
%         \hline
%         5 &
%         sometime P &
%         let CS:Snapshot = self.snapshot in Set\{CS\}->closure(s | s.succ())->exists(s | [s |= P]) \\
%         \hline
%         6 &
%         sometime P before Q &
%         let FS:Set(Snapshot) = Set\{self.snapshot\}->closure(s | s.succ())
%         in let PreS:Set(Snapshot) = Set\{self.snapshot.pred()\}->closure(s | s.pred())
%         in let AllFSQ:Set(Snapshot) = FS->select(s | [s |= Q])
%         in let FSQ:Snapshot = AllFSQ->any(s | Set\{s\}->closure(s | s.succ())->includesAll(AllFSQ))
%         in let FSP:Set(Snapshot) = FS->select(s | [s |= P])
%         in if FSQ.isDefined() then (if (FSP->size() > 0) then ((Set\{FSQ.pred()\}->closure(s | s.pred())-PreS)->exists(s\_1 | FSP->includes(s\_1))) else false endif) else false endif \\
%         \hline
%         7 &
%         sometime P since Q &
%         let CS:Snapshot = self.snapshot
%         in let PS:Set(Snapshot) = Set\{CS.pred()\}->closure(s | s.pred())
%         in let AllPSQ:Set(Snapshot) = PS->select(s | [s |= Q])
%         in let PSQ:Snapshot = AllPSQ->any(s | Set\{s\}->closure(s | s.pred())->includesAll(AllPSQ))
%         in let PSP:Set(Snapshot) = PS->select(s | [s |= P])
%         in if PSQ.isDefined() then (Set{PSQ}->closure(s | s.succ())->excluding(null)->intersection(PS)->exists(s | PSP->includes(s))) else false endif \\ 
%         \hline
%         8 &
%         previous P &
%         let previousSnapshot:Snapshot = self.snapshot.pred() in [previousSnapshot |= P] \\
%         \hline
%         9 &
%         sometimePast P &
%         let CS:Snapshot = self.snapshot in Set\{CS.pred()\}->closure(s | s.pred())->exists(s | [s |= P]) \\
%         \hline
%         10 &
%         alwaysPast P &
%         let CS:Snapshot = self.snapshot in Set\{CS.pred()\}->closure(s | s.pred())->forAll(s | [s |= P]) \\
%         \hline
%         11 &
%         isCalled(Op()) [at most | at least | ] n times &
%         [OpClassName].allInstances().select(op | op.succ() = [ContextSnapshot])->size() [<= | >= | =] n \\
%         \hline
%         12 &
%         isCalled(Op($param_1$, $\ldots$, $param_n$)) [at most | at least | ] n times &
%         [OpClassName].allInstances().select(op | op.succ() = [ContextSnapshot] and
%         (Set\{op.$param_1$.succ\}->closure(p | p.succ)->includes($param_1$) or Set\{op.$param_1$.pred\}->closure(p | p.pred)->includes($param_1$)) 
%         and ($\ldots$) 
%         and (Set\{op.$param_n$.succ\}->closure(p | p.succ)->includes($param_n$) or Set\{op.$param_n$.pred\}->closure(p | p.pred)->includes($param_n$)))->size() [<= | >= | =] n \\
%         \hline
%         13 &
%         becomesTrue(P) &
%         let currentObject = self.snapshot.[ContextObject]->any(o | o.id = self.id) in
%         let previousObject = self.snapshot.pred().[ContextObject]->any(o | o.id = self.id) in 
%         not [previousObject |= P] and [currentObject |= P] \\
%         \hline
%     \end{tabularx}
%     \end{table}
