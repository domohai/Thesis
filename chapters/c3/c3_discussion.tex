\section{Discussion}

\hspace{1cm} Our TOCL+ approach for specifying and verifying temporal properties in 
UML/OCL models offers several advantages over existing approaches but also comes with 
certain limitations that warrant discussion.

The primary strength of our approach lies in its integration with standard modeling 
environments and tools. By extending OCL rather than introducing an entirely new 
formalism, we maintain compatibility with existing modeling practices while adding 
temporal verification capabilities. This integration allows modelers to work within 
familiar environments while gaining the ability to verify a broader class of 
properties. Our transformation-based approach also offers considerable flexibility. 
By converting TOCL+ specifications to standard OCL constraints on filmstrip models, 
we leverage established OCL tool capabilities without requiring specialized temporal 
verification engines. This provides a pragmatic solution that can be adopted without 
significant infrastructure changes or learning costs. The event-based constructs in 
TOCL+ address a significant gap in existing temporal OCL extensions. While previous 
approaches like TOCL provide temporal operators, they lack robust support for event 
detection and bounded existence properties. Our extensions make it possible to express 
important constraints related to operation calls and state changes, significantly 
broadening the range of verifiable properties.

While effective in practice, our approach has several technical limitations. The OCL 
constraints generated by our transformation can be complex, potentially affecting 
verification performance for large models with numerous temporal properties. These 
constraints involve intricate navigation through snapshots and careful handling of 
object identity, which may impact scalability for very large systems. Our approach 
also requires modelers to add identifier attributes to domain classes to maintain 
object identity across snapshots. This requirement introduces a small burden on 
modelers and slightly modifies the original domain model. A more elegant solution 
would be to handle object identity tracking automatically within the transformation 
framework. The current implementation also has limited support for complex expressions 
within event specifications. For example, nested event expressions or complex guards 
on events might not translate correctly in all cases. This restricts the sophistication 
of temporal properties that can be reliably verified.

From a methodological perspective, our work lacks a formal definition of the TOCL+ 
language and its transformation to OCL. Although our patterns appear to work correctly 
based on empirical evidence from case studies, we haven't provided formal proofs of 
semantic preservation between TOCL+ expressions and their OCL translations. This 
represents a theoretical gap that could affect confidence in the correctness of 
verification results for complex scenarios. Additionally, while our approach handles 
the core temporal operators and event constructs well, it doesn't yet support 
probabilistic or real-time properties. Systems with strict timing requirements or 
probabilistic behavior patterns would require extensions beyond the current 
capabilities.

Several promising research directions emerge from these limitations. The development 
of optimization techniques for the generated OCL constraints could significantly 
improve verification performance. Extending the approach with quantitative time 
constraints would enable verification of real-time properties. Creating a pattern 
library of common temporal verification scenarios with optimized translations would 
make the approach more accessible to practitioners. Enhancing the plugin with more 
intuitive visualizations of counterexamples when temporal properties are violated 
would improve usability. The formal definition of TOCL+ semantics and transformation 
rules, along with mathematical proofs of their correctness, would strengthen the 
theoretical foundations of the approach and provide stronger guarantees about 
verification results.