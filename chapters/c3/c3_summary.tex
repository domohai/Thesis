\section{Summary}
\hspace{1cm} In this chapter, we presented the practical implementation of our TOCL+ 
approach through a plugin for the USE tool and demonstrated its effectiveness through 
a detailed case study. The implementation bridges the gap between the theoretical 
foundations established in Chapter 2 and practical model verification by enabling 
modelers to specify temporal properties in a high-level language while leveraging 
existing OCL verification tools. Our plugin successfully implements the transformation 
rules that convert TOCL+ expressions into equivalent OCL constraints interpreted 
over filmstrip models. The implementation uses ANTLR4 for parsing TOCL+ expressions 
and employs a listener-based approach to systematically translate temporal operators 
and event constructs into structural OCL navigations.

The Software System case study demonstrated how our approach enables verification 
of diverse temporal properties that would be impractical to express in standard OCL. 
The safety properties successfully enforced correct operational sequencing and 
uniqueness constraints, while the liveness property verified eventual progress in 
the system. The complexity of the generated OCL constraints highlights the value of 
our approach: while these constraints involve intricate navigation through snapshots 
and require careful handling of object identity, the TOCL+ specification remains 
simple and intuitive.

Our discussion has identified both strengths and limitations of the approach. 
The primary strengths include seamless integration with standard modeling environments,
flexibility through the transformation to standard OCL, and enhanced expressiveness 
through event constructs. However, technical limitations such as complex generated 
constraints, object identity management requirements, and methodological limitations 
including the lack of formal definitions present opportunities for improvement. 
We have also outlined promising research directions including optimization techniques, 
support for real-time properties, and formal definition of the transformation semantics.

While our experiments provide empirical evidence supporting the correctness of the 
transformation rules, it's important to note that we have not formally proven 
the correctness of these translations mathematically. Nevertheless, the consistent 
verification results across different temporal properties and scenarios provide 
confidence in the practical applicability of the approach to realistic modeling 
scenarios. The verification performance remained acceptable despite the increased 
complexity of the generated constraints. This chapter thus validates our approach 
experimentally, showing how temporal verification can be effectively integrated 
into standard UML/OCL modeling environments without requiring specialized temporal 
verification tools.
% \section{Summary}
% \hspace{1cm} In this chapter, we presented the practical implementation of our TOCL+ 
% approach through a plugin for the USE tool and demonstrated its effectiveness through 
% a detailed case study. The implementation bridges the gap between the theoretical 
% foundations established in Chapter 2 and practical model verification by enabling 
% modelers to specify temporal properties in a high-level language while leveraging 
% existing OCL verification tools. Our plugin successfully implements the transformation 
% rules that convert TOCL+ expressions into equivalent OCL constraints interpreted 
% over filmstrip models. The implementation uses ANTLR4 for parsing TOCL+ expressions 
% and employs a listener-based approach to systematically translate temporal operators 
% and event constructs into structural OCL navigations.

% The Software System case study demonstrated how our approach enables verification 
% of diverse temporal properties that would be impractical to express in standard OCL. 
% The safety properties successfully enforced correct operational sequencing and 
% uniqueness constraints, while the liveness property verified eventual progress in 
% the system. The complexity of the generated OCL constraints highlights the value of 
% our approach: while these constraints involve intricate navigation through snapshots 
% and require careful handling of object identity, the TOCL+ specification remains 
% simple and intuitive. While our experiments provide empirical evidence supporting 
% the correctness of the transformation rules, it's important to note that we have not 
% formally proven the correctness of these translations mathematically. Nevertheless, 
% the consistent verification results across different temporal properties and scenarios 
% provide confidence in the practical applicability of the approach to realistic 
% modeling scenarios. The verification performance remained acceptable despite the 
% increased complexity of the generated constraints. This chapter thus validates our 
% approach experimentally, showing how temporal verification can be effectively integrated 
% into standard UML/OCL modeling environments without requiring specialized temporal 
% verification tools.
% % \hspace{1cm} In this chapter, we presented the practical implementation of our TOCL+ 
% % approach through a plugin for the USE tool and demonstrated its effectiveness through 
% % a detailed case study. The implementation bridges the gap between the theoretical 
% % foundations established in Chapter 2 and practical model verification by enabling 
% % modelers to specify temporal properties in a high-level language while leveraging 
% % existing OCL verification tools. Our plugin successfully implements the transformation 
% % rules that convert TOCL+ expressions into equivalent OCL constraints interpreted 
% % over filmstrip models. The implementation uses ANTLR4 for parsing TOCL+ expressions 
% % and employs a listener-based approach to systematically translate temporal operators 
% % and event constructs into structural OCL navigations.

% % The Software System case study demonstrated how our approach enables verification 
% % of diverse temporal properties that would be impractical to express in standard OCL. 
% % The safety properties successfully enforced correct operational sequencing and 
% % uniqueness constraints, while the liveness property verified eventual progress in 
% % the system. The complexity of the generated OCL constraints highlights the value of 
% % our approach: while these constraints involve intricate navigation through snapshots 
% % and require careful handling of object identity, the TOCL+ specification remains 
% % simple and intuitive. Our experiments confirm both the correctness of the 
% % transformation rules and the practical applicability of the approach to realistic 
% % modeling scenarios. The verification performance remained acceptable despite the 
% % increased complexity of the generated constraints. This chapter thus validates our 
% % approach not just theoretically but in practice, showing how temporal verification 
% % can be effectively integrated into standard UML/OCL modeling environments without 
% % requiring specialized temporal verification tools.