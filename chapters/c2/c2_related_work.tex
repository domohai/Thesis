\section{Related Work}
\hspace{1cm} Several approaches have addressed the challenge of specifying and verifying temporal 
properties in UML models. Al Lail et al. \cite{TPV} proposed a technique similar to 
ours, transforming application models into Snapshot Transition Models (STM) and using 
TOCL for specification, but their approach inherits TOCL's limitations with 
event-based properties. Kanso and Taha \cite{TOCL_Taha} and Dadeau et al. 
\cite{verification_TOCL_Taha} explored pattern-based specification approaches, 
implementing patterns in Eclipse but without providing verification capabilities, 
or focusing on model-based testing rather than comprehensive verification. Baresi 
et al. \cite{MADES} present the MADES approach, which combines several heavyweight 
formalisms for verifying embedded systems, but requires steep learning curves and 
mathematical skills that make it inaccessible for many UML designers. Similarly, 
Combemale et al. \cite{Combemale_related_work_2} use a temporal extension to OCL 
with translation to Petri nets, requiring designers to learn additional formalisms 
to understand verification results. Meyers et al. \cite{ProMoBox_related_work_3} 
support verifying temporal properties in domain-specific modeling through a family 
of five languages, but rely on LTL and the Spin model checker, creating a separation 
between specification and verification languages. Hilken and Gogolla 
\cite{verifying_filmstripping} presented a filmstripping approach similar to our 
verification technique, but without support for high-level temporal property 
specification. In contrast, our approach unifies specification and verification 
within the familiar OCL context, eliminating the need to learn additional formalisms.
% 1
% \cite{TPV} is an approach similar to the technique used in our work. The approach 
% transforms application model into a so-called Snapshot Transition Model (STM) which 
% and use TOCL to specify temporal properties. But the approach contains limitations of 
% the TOCL extension mentioned above.

% 2
% The research work described in \cite{TOCL_Taha} \cite{verification_TOCL_Taha} discusses
% pattern-based specification approaches. \cite{TOCL_Taha} implements the patterns on 
% top of Eclipse but does not provide a verification tool. On the other hand, \cite{verification_TOCL_Taha}
% describes a new property, a model-based testing approach using UML/OCL models to 
% evaluate the quality of test suites. 

% 3
% In \cite{MADES}, the MADES approach is presented, which combines several heavy
% weight formalisms and techniques to verify embedded systems.
%  These techniques require steep learning and mathematical skills,
%  makingthetoolhardtousebyUMLdesigners.

% 4
% Combemale et al. \cite{Combemale_related_work_2} use a temporal extension to OCL based 
% on process states to specify temporal constraints. These constraints
% are then translated to Petri nets for verification. Designers are,
% therefore, required to learn Petri nets to understand the verification results.

% 5
% Similarly, ProMoBox \cite{ProMoBox_related_work_3} supports verifying temporal
%  properties in the context of domain-specific modeling. ProMoBox
%  defines a family of five languages that are required to support property
% specification and verification. Properties are specified in LTL,
%  and models are translated to the Spin model checker for verification.

% 6
% In \cite{verifying_filmstripping}, an approach to verify linear temporal logic 
% properties similar to our verification approach is presented. But this approach
% does not support specification of temporal properties.