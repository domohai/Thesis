\section{An Extended OCL for Temporal and Event Specifications}

%%%%% TOCL
%%% TODO: Improve this section
\subsection{TOCL}

\hspace{1cm} In this thesis, we leverage TOCL, as introduced by Ziemann and Gogolla \cite{TOCL}, 
as the temporal foundation for specifying properties that must hold over time across 
multiple states of a system. Standard Object Constraint Language (OCL) is limited to 
evaluating constraints within a single system state or across a single state transition 
(via pre- and postconditions), which is insufficient for capturing the dynamic behaviors 
inherent in many system requirements. For instance, properties such as "eventually, 
the system will reach a stable state" or "once a condition is met, it must remain 
true thereafter" require reasoning over sequences of states. TOCL addresses this limitation 
by extending OCL with elements of linear temporal logic, enabling the expression of such 
temporal properties within a familiar OCL-like syntax.

TOCL's comprehensive set of temporal operators, categorized into future and past 
operators, provides the essential temporal reasoning capabilities for our work. 
In this thesis, we adopt these operators unchanged as the basis for modeling and 
verifying dynamic system behaviors over time. However, to address systems that 
exhibit reactive behaviors driven by specific events, we extend TOCL into TOCL+ 
by integrating novel event-based constructs. This extension, detailed in the next 
section, complements TOCL’s temporal framework, enabling a more holistic specification 
of both state-based temporal properties and event-driven dynamics. Below, we review 
the adopted TOCL temporal operators, their syntax, and semantics, which serve as 
the cornerstone of TOCL+.

%%% Version 1
% TOCL enhances the Object Constraint Language (OCL) by enabling the 
% specification of properties that must hold over time, across multiple states of 
% a system. While standard OCL is limited to evaluating constraints within a 
% single system state or across a single state transition (via pre- and postconditions), 
% many system requirements involve dynamic behaviors that unfold over sequences of states. 
% Examples include properties such as "eventually, the system will reach a stable state" 
% or "once a condition is met, it must remain true thereafter." To address this, 
% TOCL, as introduced by Ziemann and Gogolla \cite{TOCL}, extends OCL with 
% elements of linear temporal logic, allowing these temporal properties to be expressed 
% directly within a familiar OCL-like syntax.

% TOCL introduces a comprehensive set of temporal operators, divided into future 
% and past categories, which are adopted in TOCL+ as the foundation for temporal 
% reasoning. Below, we review these operators, their syntax, semantics, and provide 
% illustrative examples.

\subsubsection{Adopted TOCL Temporal Operators}
The temporal operators in TOCL are categorized as follows:

\paragraph{Future Operators:}
\begin{itemize}
    \item \textbf{next $e$:} True if the expression $e$ holds in the next state.
    
    \item \textbf{always $e$:} True if $e$ holds in the current state and all subsequent states.
    
    \item \textbf{sometime $e$:} True if $e$ holds in the current state or at least one future state.
    
    \item \textbf{always $e_1$ until $e_2$:} True if $e_1$ remains true until $e_2$ becomes true, or if $e_1$ remains true indefinitely if $e_2$ never becomes true.
    
    \item \textbf{sometime $e_1$ before $e_2$:} True if $e_1$ becomes true at some point before $e_2$ does, or if $e_1$ becomes true and $e_2$ never does.
\end{itemize}

\paragraph{Past Operators:}
\begin{itemize}
    \item \textbf{previous $e$:} True if $e$ was true in the previous state (or if there is no previous state, i.e., at the initial state).
    
    \item \textbf{alwaysPast $e$:} True if $e$ was true in all past states.
    
    \item \textbf{sometimePast $e$:} True if $e$ was true in at least one past state.
    
    \item \textbf{always $e_1$ since $e_2$:} True if $e_1$ has been true since the last time $e_2$ was true.
    
    \item \textbf{sometime $e_1$ since $e_2$:} True if $e_1$ has been true at some point since the last time $e_2$ was true.
\end{itemize}

These operators enable precise specification of temporal relationships, making TOCL 
a critical component of our extended framework, TOCL+. In the following section, we 
present the formal syntax and semantics of these temporal operators to provide a 
complete understanding of their application within our approach.

%%% Version 1
% These operators enable precise specification of temporal relationships, making TOCL suitable for modeling and verifying dynamic system behaviors.

%%% Sample
% Future Operators:

% next e: True if the expression e holds in the next state.
% always e: True if e holds in the current state and all subsequent states.
% sometime e: True if e holds in the current state or at least one future state.
% always e1 until e2: True if e1 remains true until e2 becomes true, or if e1 remains true indefinitely if e2 never becomes true.
% sometime e1 before e2: True if e1 becomes true at some point before e2 does, or if e1 becomes true and e2 never does.

% Past Operators:

% previous e: True if e was true in the previous state (or if there is no previous state, i.e., at the initial state).
% alwaysPast e: True if e was true in all past states.
% sometimePast e: True if e was true in at least one past state.
% always e1 since e2: True if e1 has been true since the last time e2 was true.
% sometime e1 since e2: True if e1 has been true at some point since the last time e2 was true.

%
\subsubsection{Syntax and Semantics}
The syntax of TOCL integrates these temporal operators seamlessly into OCL expressions, allowing them to be used within invariants, preconditions, and postconditions. For example:

An invariant using \textit{always} operator:
\begin{lstlisting}[style=toclstyle]
context C inv: always (self.attribute > 0)
\end{lstlisting}
% context C inv:
%   always (self.attribute > 0)

A condition using \textit{next} operator:
\begin{lstlisting}[style=toclstyle]
context C inv:
  self.state = #active implies next (self.state = #idle)
\end{lstlisting}
% context C inv:
%   (self.state = \#active) implies next (self.state = \#idle)

The semantics of these operators are defined over infinite sequences of system states, where each state represents a snapshot of the system at a given time. The evaluation of an expression depends on its position within this sequence:

\begin{itemize}
  \item \textbf{next $e$:} True if $e$ holds at the state immediately following the current one.
  \item \textbf{always $e$:} True if $e$ holds at the current state and all future states.
  \item \textbf{sometime $e$:} True if $e$ holds at the current state or some future state.
  \item \textbf{For past operators:} The evaluation considers the sequence of states preceding the current state, with \textit{previous $e$} being true if $e$ held in the prior state, and so forth.
\end{itemize}
% next e is true if e holds at the state immediately following the current one.
% always e is true if e holds at the current state and all future states.
% sometime e is true if e holds at the current state or some future state.
% For past operators, the evaluation considers the sequence of states preceding the current state, with previous e being true if e held in the prior state, and so forth.

Formal definitions of the semantics are provided in [28], based on a state sequence ($\hat{\sigma} = \langle \sigma_0, \sigma_1, \ldots \rangle$), ensuring a rigorous foundation for TOCL. For a detailed formal treatment, readers are referred to the original paper.

%
\subsubsection{Example Specifications}
To demonstrate the practical application of these operators, we adapt examples 


These examples highlight how TOCL's temporal operators enable the specification of 
complex dynamic properties, forming a critical component of the TOCL+ language. 
In the subsequent subsections, we build upon this foundation by introducing 
event-based constructs and their integration with these temporal capabilities.

%%%%% Event
\subsection{Event Constructs in OCL}

\hspace{1cm} Events are predicates that specify sets of instants within a system's 
timeline. In object-oriented systems, several types of events can be observed: 
operation events (call/start/end), time-triggered events, and state change events. 
Since time-triggered events can be considered particular cases of state change events 
when a clock is integrated into the system, our extension focuses on operation events 
and state change events.

Our approach to event specification adopts the synchronous paradigm, which provides 
well-founded mathematical semantics and enables formal verification. The essence 
of this paradigm is the atomicity of reactions (operation calls) where all occurring 
events during a reaction are considered simultaneous. Under this paradigm, an 
operation call leads the system directly from a pre-state to a post-state without 
intermediate states being observable.

To capture these concepts, TOCL+ introduces two primary event constructs:

\begin{itemize}
    \item \textbf{isCalled}: A generic event construct that unifies operation events. It detects when an operation is invoked on an object, representing the atomic transition from a pre-state to a post-state. This construct can be refined with optional pre-state and post-state conditions.
    
    \item \textbf{becomesTrue}: A state change event that is parameterized by an OCL boolean expression P. It designates a step in which P becomes true (i.e., P was evaluated to false in the previous state and is true in the current state). In the object-oriented paradigm, a state change is necessarily a consequence of some operation call, therefore becomesTrue acts as syntactic sugar for any operation call that causes P to switch from false to true.
\end{itemize}

\subsubsection{Formal Definition}

Formally, we define events in terms of operations and state transitions. Let O be the set of all operations and E be the set of all OCL boolean expressions in a model. An event is either:

\begin{itemize}
    \item isCalled(op) - representing a call to operation op, optionally with precondition pre and postcondition post
    \item becomesTrue(P) - representing any operation call that transitions the system from a state where ¬P holds to a state where P holds
\end{itemize}

This formal definition enables precise reasoning about when events occur during system execution and forms the foundation for our verification approach.

\subsubsection{Examples}

To illustrate these event constructs, consider a banking system with an Account class:

\begin{itemize}
    \item isCalled(account.withdraw(100)) - Detects when the withdraw operation is called with parameter 100 on an account object
    \item becomesTrue(account.balance < 0) - Detects when an account's balance transitions from non-negative to negative
\end{itemize}

These examples demonstrate how event constructs enable the specification of critical moments in a system's execution, providing the foundation for more complex temporal properties. By combining these event constructs with the temporal operators from TOCL, we create a powerful specification language capable of expressing reactive behaviors and complex temporal patterns.

%%% Sample
% Events are predicates to specify sets of instants within the time line. In Section 1.4.2, we discussed the different types of
% events in the object-oriented approach. There are operation (call/start/end) events, time-triggered events and state change
% events. We have seen that when integrating the clock into the system, time-triggered events are particular state change
% events. Hence, we only need to extend OCL with the necessary construct for both operation and state change events.
% We aim to connect our OCL temporal extension to formal methods such as model-checking and test scenario generation.
% Formal methods are mainly based on the synchronous paradigm that has well-founded mathematical semantics and that
% allows formal verification of the programs and automatic code generation. The essence of the synchronous paradigm is the
% atomicity of reactions (operation calls) where all the occurring events during such a reaction are considered simultaneous.
% In our work, we adopt the synchronous paradigm, and we merge the operation (call/start/end) events into one call event,
% named isCalled, that leads the system from a pre-state to a post-state without considering neither observing intermediate
% change states.
% isCalled: is a generic event construct that unifies both operation events and state change events.
% becomesTrue: is a state change event that is parameterized by an OCL boolean expression P, and designates a step in
%  which P becomes true, i.e. P was evaluated to false in the previous state. In the object-oriented paradigm, a state change
%  is necessarily a consequence of some operation call, therefore the becomesTrue construct is a syntactic sugar and stands for
%  any operation call switching P to true.
% Formal semantics
% A test case is a scenario in which a set of operations is executed in sequence with observations either before or after each
%  operation execution. Since we are interested in test cases generation (see Section 10), we adopt a scenario-based semantics
%  over the synchronous paradigm to formalize our temporal extension. The essence of that paradigm is the atomicity of
%  reactions (operation calls) where all the events occurring during such a reaction are considered as simultaneous. A reaction
%  is one atomic call event, that leads the system directly from a pre-state to a post-state without going through intermediate
%  states.
% We define the set of all atomic events of a given object model as follows:
% Definition6.1 (Alphabetofatomicevents). Let O be the set of all operations and E be the set of all OCL boolean expressions of
%  an object model M.ThealphabetΣM ofatomicevents (abbreviated as Σ in the following) is defined by the set O×E×E.
%  An atomic event e ∈Σ takes the form: e =(op). It stands for a call of the operation op in a context where pre
%  stands for the precondition satisfied in the pre-state and post stands for the postcondition satisfied in the post-state.
%  We now give the formal meaning of the notion of events introduced in our OCL temporal extension.
%  Definition 6.2 (Events). Let Σ be the alphabet of atomic events, O be the set of all operations and E the set of all OCL
%  boolean expressions. An event is either an isCalled(op,pre,post) or a becomesTrue(P) with:
%  isCalled(op,pre,post) = op,pre,post ∈ Σ pre⇒pre, post⇒post and
%  becomesTrue(P) = (op,pre,post) ∈ Σ op∈O, pre⇒¬P, post⇒ P
%  Definition 6.2 calls for the following comments:– An event does not represent a single atomic event, but a specific subset of atomic events. It is intuitively the set of all
%  atomic events in which the operation op is invoked, in a pre-state which implies the expression pre and leading to a
%  post-state which implies the expression post. The set of all events is then defined6 as the set 2Σ;
