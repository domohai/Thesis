% \hspace{1cm} Figure \ref{sec:verification_approach} presents a general overview of the approach.
% First, the UML modeler creates an Application Model and defines temporal properties
% in TOCL+. The first step is to convert the dynamic specification of the Application
% Model into static ones by transforming it into a filmstrip model. The filmstrip model
% is a class model that allows the representation of dynamic behaviors of a system in
% terms of snapshots and operation calls. The second step of the approach translates 
% a TOCL+ property into an OCL expression interpreted on a filmtrip model. The TOCL+
% properties are translated into OCL query expressions and constraints, which traverse
% the snapshots and operation calls of the filmstrip model, validating the correctness
% of all snapshots given the specified properties. Given (1) a class model representing 
% the behavior and (2) an OCL expression constraining this behavior, we can do 
% the verification using a UML static-analysis tool, which is the third step of the 
% approach. Our approach uses the USE Model Validator \cite{USE_Validator} to perform
% analysis on the model. The USE Model Validator also requires a configuration file 
% that defines the pre-configured search space. The model validator then looks for a
% solution that meets the requirements of the given search space. If there is a solution,
% the model validator gives the output showing SATISFIABLE. If there is no valid state
% that satisfies all the conditions, UNSATISFIABLE is reported by the validator.

% As depited in the Figure \ref{sec:verification_approach}, the approach consists of
% three main steps that are explained in more details using the Software System example
% in Figure \ref{fig:class_diagram_software_system}.


%%% Step 1
%%% Version 1
% \subsection{Step 1: Transforming the Application Model into a Filmstrip Model}
% At this step, we convert the dynamic specification of the Application Model into a
% static representation using a Filmstrip Model introduced in \ref{subsec:filmstripping}.
% We will describe the transformed filmstrip model based on the transformation rules
% specified in \ref{subsec:filmstripping}.

% The two classes \texttt{System}, \texttt{Application}
% and their attributes are the same after transformed into filmstrip model. Furthermore,
% \texttt{Snapshot} and \texttt{OperationCall} classes are added. 
% \texttt{OperationCall} is an abstract class that represents the operation calls that
% modify the state of the system model from one snapshot to another. Therefore, the 
% \texttt{OperationCall} class has two associations with the \texttt{Snapshot} class
% to represent the before and after snapshots of the operation calls. The operations
% in the Application Model are specified as subclasses of the \texttt{OperationCall} 
% class. To illustrate, there are four operations in the Software System model
% (load(app : Application), install(), run(app : Application), stop(app : Application)).
% Therefore, four subclasses are created (load\_SystemOpC, install\_SystemOpC, 
% run\_SystemOpC, stop\_SystemOpC) to represent the operations. Objects of these
% \texttt{OperationCall} subclasses represent specific calls of the operations that
% the subclasses derived from. Each subclass will have a number of attirbutes. 
% An aSelf attribute which refers to the object that the operation is invoked on which
% is at the before state.
% The parameter attributes are the parameters of the operation. The return value that
% the operation returns after the call.
% Transformation of assocations and invariants are applied without any changes. 
% It is worth noting that the transformation of pre and post conditions. 
% Pre and post conditions are transformed into invariants defined on the operation
% call subclasses. These invariants ensure that the operation call subclasses are 
% only instantiated when the operation pre and post conditions are satisfied.
% To illustrate, a selection of OCL pre and post conditions for the \texttt{load} 
% operation is displayed in Listing \ref{lst:pre_post_load} and its corresponding 
% invariants are shown in Listing \ref{lst:invariants_load_SystemOpC}.
% The final filmstrip model is depicted in Figure \ref{fig:filmstrip_model}. 

%%% Version 2
% \subsection{Step 1: Transforming the Application Model into a Filmstrip Model}

% \hspace{1cm} The first step in our verification approach transforms the dynamic Application Model into a static representation using the Filmstrip Model technique introduced in Section~\ref{subsec:filmstripping}. This transformation follows the systematic rules described previously, converting dynamic behavioral specifications into structural constraints that can be verified by static analysis tools.

% For our Software System example \ref{fig:class_diagram_software_system}, the transformation preserves the original classes (\texttt{System} and \texttt{Application}) with their attributes unchanged. The transformation then introduces two essential filmstrip-specific classes: \texttt{Snapshot} and \texttt{OperationCall}. The \texttt{Snapshot} class represents individual system states, while \texttt{OperationCall} serves as an abstract base class representing operations that transition the system from one state to another.

% Each operation from the original model is transformed into a concrete subclass of \texttt{OperationCall}. For our Software System with four operations—\texttt{load(app:~Application)}, \texttt{install()}, \texttt{run(app:~Application)}, and \texttt{stop(app:~Application)}—the transformation produces four corresponding subclasses: \texttt{load\_SystemOpC}, \texttt{install\_SystemOpC}, \texttt{run\_SystemOpC}, and \texttt{stop\_SystemOpC}. Each subclass contains:

% \begin{itemize}
%     \item An \texttt{aSelf} attribute referencing the object on which the operation is invoked (in its pre-operation state)
%     \item Attributes corresponding to the operation's parameters (e.g., \texttt{app} for the \texttt{load} operation)
%     \item An attribute for any return value the operation might produce
% \end{itemize}

% The \texttt{OperationCall} class establishes relationships with the \texttt{Snapshot} class through two associations representing the pre-operation and post-operation system states. These associations enable navigation between consecutive snapshots through operation calls, forming the backbone of the filmstrip structure.

% While associations and invariants from the original model are preserved without modification, the transformation of pre- and postconditions is particularly significant. These dynamic operation contracts are converted into static invariants defined on the operation call subclasses. These invariants ensure that operation call objects can only exist when both the pre- and postconditions of the corresponding operation are satisfied.

% Listing~\ref{lst:pre_post_load} shows a selection of OCL pre- and postconditions for the \texttt{load} operation in the original model. Listing~\ref{lst:invariants_load_SystemOpC} displays how these conditions are transformed into invariants in the filmstrip model, illustrating the conversion process. For example, the postcondition \texttt{loaded} is transformed into an invariant that navigates between the pre- and post-operation snapshots using the \texttt{succSystem} and \texttt{predSystem} associations.

% The resulting filmstrip model, illustrated in Figure~\ref{fig:filmstrip_model}, provides a static structural representation that captures the dynamic behavior of the original application model. This transformation is the foundation that enables the verification of temporal properties using static analysis techniques in the subsequent steps of our approach.

%%% Step 2
%%% Version 1
% In this step, the Software System class model is transformed into a sequence of 
% snapshots and operation calls represented by the Filmstrip Model in order to express 
% TOCL+ properties as OCL constraints. We first specify the temporal properties in the
% Software System model using TOCL+. Listing \ref{lst:tocl+} shows some of the Software
% System TOCL+ properties. These TOCL+ properties are specified in the context of the
% Software System class model. Then, OCL constraints are systematically produced in
% the context of the Software System filmstrip model (see Figure \ref{fig:filmstrip_model})
% using these TOCL+ properties. Each OCL constraint captures the semantics of the 
% corresponding TOCL+ property in the context of the Filmstrip Model. Listing \ref{lst:ocl_liveness}
% shows the OCL translation for liveness property defined in \ref{lst:tocl+}. The full
% OCL translations for all TOCL+ properties are provided in Appendix A.