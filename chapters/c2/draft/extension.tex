%%% TOCL
%%% Version 1
% TOCL enhances the Object Constraint Language (OCL) by enabling the 
% specification of properties that must hold over time, across multiple states of 
% a system. While standard OCL is limited to evaluating constraints within a 
% single system state or across a single state transition (via pre- and postconditions), 
% many system requirements involve dynamic behaviors that unfold over sequences of states. 
% Examples include properties such as "eventually, the system will reach a stable state" 
% or "once a condition is met, it must remain true thereafter." To address this, 
% TOCL, as introduced by Ziemann and Gogolla \cite{TOCL}, extends OCL with 
% elements of linear temporal logic, allowing these temporal properties to be expressed 
% directly within a familiar OCL-like syntax.

% TOCL introduces a comprehensive set of temporal operators, divided into future 
% and past categories, which are adopted in TOCL+ as the foundation for temporal 
% reasoning. Below, we review these operators, their syntax, semantics, and provide 
% illustrative examples.

%%% Version 1
% These operators enable precise specification of temporal relationships, making TOCL suitable for modeling and verifying dynamic system behaviors.

%%% Sample
% Future Operators:

% next e: True if the expression e holds in the next state.
% always e: True if e holds in the current state and all subsequent states.
% sometime e: True if e holds in the current state or at least one future state.
% always e1 until e2: True if e1 remains true until e2 becomes true, or if e1 remains true indefinitely if e2 never becomes true.
% sometime e1 before e2: True if e1 becomes true at some point before e2 does, or if e1 becomes true and e2 never does.

% Past Operators:

% previous e: True if e was true in the previous state (or if there is no previous state, i.e., at the initial state).
% alwaysPast e: True if e was true in all past states.
% sometimePast e: True if e was true in at least one past state.
% always e1 since e2: True if e1 has been true since the last time e2 was true.
% sometime e1 since e2: True if e1 has been true at some point since the last time e2 was true.

%
% \subsubsection{Syntax and Semantics}
% The original syntax of TOCL in [29] was defined using mathematical notations as opposed to EBNF.
% author = {Lail, Mustafa Al and Rosales, Antonio and Cardenas, Hector and Hamann, Lars and Perez, Alfredo},
% created EBNF grammar of TOCL in \cite{TOCL2OCL}. We also adopted this EBNF grammar to define the syntax of TOCL in our work.
% The syntax of TOCL integrates these temporal operators seamlessly into OCL expressions, allowing them to be used within invariants, preconditions, and postconditions. For example:

% An invariant using \textit{always} operator:
% \begin{lstlisting}[style=toclstyle]
% context C inv: always (self.attribute > 0)
% \end{lstlisting}
% context C inv:
%   always (self.attribute > 0)

% A condition using \textit{next} operator:
% \begin{lstlisting}[style=toclstyle]
% context C inv:
%   self.state = #active implies next (self.state = #idle)
% \end{lstlisting}

% The semantics of these operators are defined over sequences of system states, with 
% formal definitions provided by Ziemann and Gogolla \cite{TOCL} based on state 
% sequences ($\hat{\sigma} = \langle \sigma_0, \sigma_1, \ldots \rangle$). For the 
% complete formal treatment, readers are referred to the original research.

% The semantics of these operators are defined over infinite sequences of system states, where each state represents a snapshot of the system at a given time. 
% Formal definitions of the semantics are provided in [28], based on a state sequence ($\hat{\sigma} = \langle \sigma_0, \sigma_1, \ldots \rangle$), ensuring a rigorous foundation for TOCL. For a detailed formal treatment, readers are referred to the original paper.
% The evaluation of an expression depends on its position within this sequence:

% \begin{itemize}
%   \item \textbf{next $e$:} True if $e$ holds at the state immediately following the current one.
%   \item \textbf{always $e$:} True if $e$ holds at the current state and all future states.
%   \item \textbf{sometime $e$:} True if $e$ holds at the current state or some future state.
%   \item \textbf{For past operators:} The evaluation considers the sequence of states preceding the current state, with \textit{previous $e$} being true if $e$ held in the prior state, and so forth.
% \end{itemize}
% next e is true if e holds at the state immediately following the current one.
% always e is true if e holds at the current state and all future states.
% sometime e is true if e holds at the current state or some future state.
% For past operators, the evaluation considers the sequence of states preceding the current state, with previous e being true if e held in the prior state, and so forth.

%%% Event extension