\section{Introduction}

\hspace{1cm} OCL provides strong support for structural properties in UML models but 
falls short when specifying dynamic system behavior. Operating only on single states 
or individual transitions, OCL cannot express properties spanning multiple states or 
responding to system events. This limitation is significant for modern systems 
requiring temporal and reactive behaviors.

Temporal logics like LTL \cite{LTL} and CTL offer formal frameworks for temporal properties but 
require specialized knowledge unfamiliar to most UML designers. This creates a practical 
barrier for practitioners comfortable with UML/OCL but not with formal temporal notations.

This chapter presents two main contributions:

First, we build upon the existing Temporal OCL (TOCL) extension \cite{TOCL}, which already 
incorporates temporal operators like \textit{always}, \textit{sometime}, and 
\textit{until} for reasoning about system evolution over time. However, TOCL remains 
limited in its ability to express event-based properties. Our TOCL+ extension 
addresses this limitation by introducing enhanced event constructs for detecting 
specific system occurrences such as operation calls and state changes, along with 
support for bounded existence properties (e.g., "an operation is called exactly n 
times" or "at most n times"). TOCL+ maintains OCL's familiar syntax while 
significantly expanding its capabilities for specifying event-driven and quantified 
temporal behaviors.
% First, TOCL+ extends OCL with temporal and event capabilities. It adds temporal 
% operators like \textit{always}, \textit{sometime}, and \textit{until} for reasoning 
% about system evolution over time, and introduces event constructs for detecting 
% specific system occurrences such as operation calls and state changes. TOCL+ maintains 
% OCL's familiar syntax while enabling complex dynamic specifications.

Second, we introduce an approach that enables verification of TOCL+ 
specifications using existing tools. This approach transforms UML/OCL models into 
filmstrip models representing state sequences, and translates TOCL+ specifications 
into standard OCL constraints verifiable within these models.

% Third: Implementation of TOCL+ to OCL transformation
Third, we present our implementation approach for transforming TOCL+ expressions into standard OCL constraints. This transformation defines practical translation patterns for temporal operators and event constructs, mapping them to structural navigations through snapshots and operation calls in the filmstrip model. These translations enable the verification of complex temporal properties using existing OCL verification tools without requiring specialized temporal logic model checkers.

The chapter is organized as follows:
\begin{itemize}
    \item Section 2.2 presents the specification approach.
    
    \item Section 2.3 details the verification approach.
    
    \item Section 2.4 describes the implementation of TOCL+ to OCL transformation.
\end{itemize}

Together, these contributions provide a complete solution for both specifying and 
verifying temporal properties within the model-driven engineering paradigm.