\setlength{\parindent}{1cm}

\section{Introduction}

\hspace{1cm} This chapter presents fundamentals about concepts and artifacts essential 
to this thesis. The modeling languages such as, Unified Modeling Language (UML), 
together with Object Constraint Language (OCL), are used to describe structural and 
behavioral aspects of systems and are briefly described in this chapter. A description 
of the modeling and specification tool called UML-based Specification Environment (USE)
is presented, including its model validation capabilities that form the foundation for 
our verification approach. We explain the filmstrip model transformation process in 
detail, as it serves as the underlying mechanism for our temporal verification approach.
Additionally, we introduce Temporal OCL (TOCL) as developed in prior research by [Author et al.]. 
Their approach extends OCL with temporal operators to express properties over time and transforms 
UML and OCL models into a Snapshot Transition Model (STM) to handle dynamic behaviors. 
In their work, TOCL expressions are translated into standard OCL constraints in the context 
of the STM. We review this foundational work as it forms the theoretical basis that our 
approach builds upon, though our implementation adapts these concepts to work with filmstrip 
models rather than STM. Each of these topics forms an essential building block for understanding 
our approach to specifying and verifying temporal properties in OCL, which will be presented in 
subsequent chapters.