\setlength{\parindent}{1cm}

\section{Introduction}

\hspace{1cm} This chapter presents the fundamental concepts and tools that form the 
foundation of our approach to temporal specification and verification in Model-Driven 
Engineering (MDE). We begin with an overview of Model-Driven Engineering, which 
provides the methodological framework for our research. Within this paradigm, models 
serve as primary artifacts throughout the software development lifecycle, enabling 
rigorous analysis and verification before implementation.

This section introduces the Unified Modeling Language (UML), the industry-standard 
visual modeling language for specifying software systems. For our work, we focus 
specifically on Class Diagrams, which define a system’s abstract structure, and 
Object Diagrams, which provide concrete instances of that structure. These diagrams 
establish the formal foundation on which our temporal extensions are built.

Although UML offers rich visual notation, it lacks a formal way to express detailed 
constraints. The Object Constraint Language (OCL) fills this gap by allowing precise, 
text-based specifications that UML diagrams cannot capture graphically. We review 
OCL's core concepts and syntax, with particular attention to its strengths and 
limitations regarding temporal properties.

Finally, we explore the UML-based Specification Environment (USE), the modeling and 
verification tool that implements our approach. USE provides the infrastructure for 
defining UML models with OCL constraints and validating them. We introduce two key 
plugins that extend USE's capabilities: the Filmstrip Plugin, which implements the 
filmstripping method by transforming dynamic model checking into static verification 
through sequences of snapshots connected by operation calls; and the Model Validator 
Plugin, which enables automated analysis of models against their constraints through 
systematic state space exploration. Together, these tools form the technical foundation 
for our verification approach, enabling both the representation of temporal properties 
and their efficient verification.

Throughout this chapter, we emphasize the context and limitations of standard 
modeling approaches regarding temporal specifications and verifications, setting the stage for our 
extensions and contributions in subsequent chapters. Each section provides essential 
background knowledge required to understand our approach to specifying and verifying 
temporal properties in object-oriented systems.
