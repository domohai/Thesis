\section{Model-Driven Engineering}

Modeling in the field of software engineering is a process through which models of
 software systems are created. In traditional software development, models are used
 only as documentation or architectural descriptions of the system to be developed. In
 contrast to traditional software development, in model-driven engineering (MDE) [56],
 models are used as an abstraction of a system to deal with the growing complexity
 of large software systems and have become a central part for software development.
 In the software design phase, models can be developed using a modeling language
 that permits the creation of components of a system. In this context, the software
 systems are specified by models at an abstract level and comprise only relevant in
formation of the system. Thus, it reduces the complexity and helps to accelerate
 the design process [36] [37] and is also independent of programming languages and
 platforms. These software models can be validated and verified with the MDE tech
nologies, which can increase the quality of the system to be developed in the design
 phase. The MDE, therefore, promises a potential increase in productivity and quality
 in software development.
 The models are built using modeling languages such as the UML together with
 formal specification languages such as the OCL to describe structural and behavioral
 aspects of the system [81]. An ordinary UML/OCL application model is comprised of
 a class diagram with any number of classes, attributes, associations, and operations.
 The structural properties of the model can be described in terms of OCL invariants
 and behavioral properties in terms of operation pre- and postconditions. The invari
ants impose the possible system state to create valid object diagrams. The operation
 pre- and postconditions determine the valid system dynamics in the form of state
 transitions with intermediate operation calls. The complete application is described
 in one single model, and a transition is induced by a single operation call [36].