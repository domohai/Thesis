\section{Model-Driven Engineering}

\hspace{1cm} Modeling in software engineering involves creating abstract representations of software systems. In traditional software development approaches, models often serve merely as documentation or architectural blueprints for the system being developed. Model-Driven Engineering (MDE), by contrast, elevates models to first-class artifacts in the development process, using them as primary means to address the growing complexity of large software systems.

During the software design phase, developers create models using specialized modeling languages that facilitate the specification of system components. These models capture the system at an abstract level, including only information relevant to the design task at hand. This abstraction reduces complexity, accelerates the design process, and maintains independence from specific programming languages and implementation platforms. Furthermore, MDE technologies enable validation and verification of these models early in the development lifecycle, potentially improving the quality of the final system. Through this emphasis on modeling, MDE promises significant improvements in both productivity and quality in software development.

The models in MDE are typically constructed using standardized modeling languages such as the Unified Modeling Language (UML), often complemented by formal specification languages like the Object Constraint Language (OCL) to describe both structural and behavioral aspects of the system. A typical UML/OCL application model consists of a class diagram containing classes, attributes, associations, and operations. The structural integrity of the model is maintained through OCL invariants, while behavioral properties are specified using operation pre- and postconditions. Invariants constrain the possible system states to ensure valid object configurations, while pre- and postconditions govern valid system dynamics by defining permissible state transitions through operation calls. In this approach, the complete application is described as a cohesive model, with each transition initiated by a specific operation call.

% Modeling in the field of software engineering is a process through which models of
%  software systems are created. In traditional software development, models are used
%  only as documentation or architectural descriptions of the system to be developed. In
%  contrast to traditional software development, in model-driven engineering (MDE) [56],
%  models are used as an abstraction of a system to deal with the growing complexity
%  of large software systems and have become a central part for software development.
%  In the software design phase, models can be developed using a modeling language
%  that permits the creation of components of a system. In this context, the software
%  systems are specified by models at an abstract level and comprise only relevant in
% formation of the system. Thus, it reduces the complexity and helps to accelerate
%  the design process [36] [37] and is also independent of programming languages and
%  platforms. These software models can be validated and verified with the MDE tech
% nologies, which can increase the quality of the system to be developed in the design
%  phase. The MDE, therefore, promises a potential increase in productivity and quality
%  in software development.
%  The models are built using modeling languages such as the UML together with
%  formal specification languages such as the OCL to describe structural and behavioral
%  aspects of the system [81]. An ordinary UML/OCL application model is comprised of
%  a class diagram with any number of classes, attributes, associations, and operations.
%  The structural properties of the model can be described in terms of OCL invariants
%  and behavioral properties in terms of operation pre- and postconditions. The invari
% ants impose the possible system state to create valid object diagrams. The operation
%  pre- and postconditions determine the valid system dynamics in the form of state
%  transitions with intermediate operation calls. The complete application is described
%  in one single model, and a transition is induced by a single operation call [36].