\section{Object Constraint Language (OCL)}

\subsection{Overview}

\hspace{1cm} As explained in the previous section, UML is a graphical language for 
visualizing system structure and behavior. However, visual modeling with UML alone 
is insufficient for developing accurate and consistent software models, as UML 
diagrams cannot express all necessary constraints. The Object Management Group (OMG) 
developed the Object Constraint Language (OCL) in 1997 to address this limitation. 
OCL is a formal assertion language with precise semantics that extends UML by 
allowing developers to specify constraints that cannot be expressed graphically. 

For example, while a UML class diagram can show that a Bank has multiple Accounts, 
it cannot express that "an Account must maintain a minimum balance of 
\$100" - this requires OCL. OCL provides two kinds of descriptions: expressions 
that evaluate to values, and constraints that must evaluate to true. 
OCL is type-rich, supporting basic types (Boolean, Real, Integer, String), 
collection types (Set, Bag, Sequence, OrderedSet), and special types (tuples, 
OclAny, OclType). The language includes navigation operators to traverse model 
relationships, comprehensive collection operations, and quantifiers for building 
logical statements. OCL constraints typically appear as class invariants 
(conditions that must always be true for all instances) and operation 
pre/postconditions (conditions that must be true before and after operation execution).

%%% Sample 1
% As explained in Sect. 2.1, UML is a graphical language for visualization of the system
% state. But visual modeling with the UML alone is not enough for the development of
% accurate and consistent software model. For this reason, Object Management Group
% (OMG), in 1997, developed Object Constraint Language (OCL), which describes 
% expressions on UML models and thus extends further the functionality of UML [60].
% [For example, ... (write about limitation of UML and how OCL can help to overcome 
% it in the context of an example)]
% OCL provides two kinds of descriptions, (a) OCL expression, which is defined
%  in the context of the constrained elements and evaluates some values, and (b) OCL
%  constraint, which is restrictive statements concerning some underlying model that
%  should evaluate to true. Basically, an OCL constraint is referred to as an OCL
%  expression of the Boolean type.
% The OCLprovides variables and operations which can be combined in various ways
% to build expressions. There are many OCL expression types. The simplest types are
% the basic type which consists of the types Boolean, Real, Integer and String. The
% concrete collection types are Bag, Sequence, Set and OrderedSet. The other types
% include tuples, which can contain one to many values of other types and special types
% such as OclAny, OclType, OclExpression, and OclState.
% There are three different types of OCL constraints, namely invariants of classes,
%  preconditions and postconditions of operations, which are used to describe model
%  properties. An invariant is a restriction in the form of expression which is applied
%  to all instances of the class diagram and that must be true for all instances, i.e., an
%  invariant must be satisfied after the creation of an instance of the class for which
%  the invariant is defined.

%%% Sample 2
% OCL is a formal assertion language, easy to use, with precise and unambiguous semantics [5]. It allows the annotation
%  of any object-oriented model, even if it is most used within UML diagrams. OCL is very rich, it includes fairly complete
%  support for:
%  – Navigation operators to navigate within the object-oriented model;
%  – Set/sequence operations to manipulate sets and sequences of objects;
%  – Universal/existential quantifiers to build first order (logic) statemeents;

\subsection{OCL Limitations}

%%% Sample
