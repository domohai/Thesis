
\section{Unified Modeling Language (UML)}

\hspace{1cm} The Unified Modeling Language (UML) is a graphical language for 
visualizing, specifying, constructing, and documenting software-intensive systems.
This unified language is maintained by the Object Management Group (OMG) [58].

The UML has become one of the most widely used modeling language that can
be used with all significant object and component methods for describing real-world
application domains. Software systems, in today’s world, are growing in size, complex
ity, distribution and importance. As a result, building and maintenance of software
have become a more complex and challenging task. Therefore, the deployment of
languages such as UML reduces the complexity and difficulty by providing a high
level of abstraction, which describes precise and essential information for designing
and developing the software system [50].  

The graphical representation of the UML includes a set of diagrams, each focusing
on different aspects of a design. The UML Notation Guide states all the notation of 
diagram elements [58]. These diagrams can be classified into two groups (a) a structural
diagram that represents the static aspect of the system, and (b) a behavioral diagram
that describes the dynamic aspect of the system. Altogether, fourteen different model
types can be found in the Unified Modeling Language Reference Manual [71]. In this
thesis, three diagrams, i.e., class diagram, object diagram and sequence diagram, have
been extensively used and are explained further in this section.

\subsection{Class Diagram}

\subsection{Object Diagram}

\subsection{Sequence Diagram}