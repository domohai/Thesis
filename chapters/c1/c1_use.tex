\section{UML-based Specification Environment (USE)}

\subsection{Overview}

\subsection{USE Features}

\subsection{USE Model Validator}

%%% Sample
% In USE, a plugin called model validator has been developed and using it, a 
% developer can automatically generate different object diagrams for a class diagram in a
%  pre-configured search space [66]. The search space of the model validator is 
% configured based on the given configuration and sometimes also on the specified additional
%  invariants. The model validator uses the external tool Kodkod [78] and SAT-based
%  methods for searching the solution. Using Kodkod, the model is expressed in relational
% logic [44]. Then it represents the model expressed in relational logic as a SAT
%  problem. Finally, a SAT solver checks its validity. The first solution found is immediately
% instantiated as an object diagram and made visible by the USE, however, the
% developer can also explore the other available solutions [49].
%  The configurations are written in text files with the extension .properties. In the
%  configuration file, a lower and upper limit can be specified for each configurable base
%  type (Integer, Real, String), each class, each attribute of a class, and associations,
%  which specifies how many instances should be there in the object diagram. In addition,
%  it is possible to specify a set of possible values of the field as a set. These specifications
%  can speed up the validation process, as the model validator only needs to search for
%  object diagrams that meet the model constraints. In addition to the configuration,
%  the developer can also provide additional OCL invariants to guide the model validator
%  to generate specific scenarios in the form of object diagrams [37].
%  The validate command in the tool USE initializes the model validator to look for a
%  solution that meets the requirements of the given search space. If there is a solution,
%  the model validator gives the output showing SATISFIABLE. If there is no valid state
%  that satisfies all the conditions, UNSATISFIABLE is reported by the validator.

\subsection{Filmstripping}

%%% Sample
% In MDE, it is necessary to check that an ordinary UML/OCL model meets the formal
% and informal described requirements [42]. The UML and OCL application models
% can involve structural aspects in the form of OCL invariants and dynamic aspects
%  in the form of operation pre- and postconditions. The changes in object states are
%  triggered by operation calls. A variety of validation and verification techniques are
%  available [21] [44] [18] [78] [15] [5] [67] which usually concentrate on checking the struc
% tural aspects of the model. The model validator in the tool USE is also specifically
%  designed for structural analysis of models.
% In order to validate dynamic aspects of the model, a so-called filmstrip approach 
% has been developed and integrated in the tool USE [36]. Filmstripping transforms a
% given UML and OCL model, which is comprised of invariants and pre- and postconditions 
% into an equivalent model that possesses only invariants. The transformed model
% is called a filmstrip model. The filmstrip model is comprised of a regular application
%  model but with additional classes to supplement the system state information. This
%  means the original model is present and unchanged with its full functionality in a
%  filmstrip model. The additional classes create new objects called snapshots, which
%  represent a single system state of an application model. This makes it possible to
%  read the progression information in a single object diagram of a filmstrip model, i.e., a
%  sequence of operation calls and object diagrams of an application model corresponds
%  to a single object diagram of the filmstrip model. The pre- and postconditions of the
%  application model are transformed into invariants and removed from the operations.
%  So the complete dynamic behavior is integrated in a single structure.
%  The transformed model involves only structural aspects and can be validated with
%  the available USE model validator. So, this transformation approach validates and
%  verifies both the static and dynamic properties of any given application model.

\subsubsection{Filmstrip Model Transformation}