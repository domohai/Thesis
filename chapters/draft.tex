%%% CONCLUSION
% This thesis has addressed the challenge of specifying and verifying temporal 
% properties in UML/OCL models without requiring specialized temporal verification 
% tools. By extending OCL with temporal operators and event constructs, and providing 
% a transformation mechanism to standard OCL, we have created an approach that enables 
% modelers to verify complex behavioral properties while remaining within familiar 
% modeling environments.

% Our work has made several significant contributions to the field of model verification:

% First, we introduced TOCL+ (Temporal OCL+), an extension of OCL that incorporates 
% temporal operators (such as "always," "eventually," and "until") and event constructs 
% (like "becomesTrue" and "isCalled"). This extension enables modelers to express 
% complex temporal properties in a concise and intuitive way, addressing a significant 
% limitation of standard OCL.

% Second, we developed a transformation approach that converts TOCL+ expressions into 
% equivalent standard OCL constraints that can be verified over filmstrip models. This 
% approach leverages existing OCL tools for temporal verification without requiring 
% specialized temporal logic model checkers. The transformation systematically maps 
% temporal operators and event constructs to structural navigations through snapshots, 
% handling the complexities of object identity and state transitions.

% Third, we implemented our approach as a plugin for the USE tool, demonstrating its 
% practical applicability. The plugin uses ANTLR4 for parsing TOCL+ expressions and 
% employs a listener-based approach to generate the corresponding OCL constraints. 
% This implementation bridges the gap between the theoretical foundations and practical 
% verification tasks.

% Fourth, we validated our approach through a detailed case study of a Software System 
% model, verifying various types of temporal properties including safety, uniqueness, 
% and liveness constraints. The experimental results confirmed that our transformation 
% approach correctly preserves the semantics of temporal properties while remaining 
% practical for real-world modeling scenarios.

% \section*{Significance of the Work}

% The significance of our work lies in making temporal verification accessible within 
% standard modeling environments. By allowing modelers to specify temporal properties 
% at a high level of abstraction while handling the verification complexity behind 
% the scenes, we enable more comprehensive model verification without requiring 
% modelers to learn specialized temporal logics or verification tools.

% Our approach is particularly valuable for verifying behavioral properties that cannot 
% be expressed in standard OCL, such as correct operational sequencing, eventual 
% progress, and bounded existence constraints. These properties are essential for 
% ensuring the correctness of dynamic system behavior, especially in domains where 
% operational sequences and timing constraints are critical.

% \section*{Limitations}

% While our approach has proven effective in practice, several limitations should be 
% acknowledged. Most notably, we have not formally defined the transformation from TOCL+ to OCL, 
% nor have we provided a metamodel for the TOCL+ language extension. These would provide 
% a more rigorous foundation for our approach and enable formal analysis of the language itself.
% Similarly, we have not formally proven the correctness of our 
% transformation rules mathematically. Although empirical evidence from our experiments 
% supports their correctness, a formal proof would provide stronger guarantees about 
% the preservation of temporal semantics.

% Additionally, the OCL constraints generated by our transformation can be complex 
% and potentially impact verification performance for large models with numerous 
% temporal properties. The constraints involve intricate navigation through snapshots 
% and careful handling of object identity, which may affect scalability for very large 
% systems.

% \section*{Future Work}

% Several directions for future research emerge from our work:

% \begin{itemize}
%     \item \textbf{Formal verification:} Developing formal proofs for the correctness 
%     of our transformation rules would strengthen the theoretical foundations of our 
%     approach.

%     \item \textbf{Optimization techniques:} Exploring optimization strategies to 
%     reduce the complexity of generated OCL constraints could improve verification 
%     performance for large models.

%     \item \textbf{Pattern library:} Creating a library of common temporal verification 
%     patterns with optimized translations would make the approach more accessible to 
%     practitioners.

%     \item \textbf{Tool integration:} Extending our approach to other UML modeling 
%     tools beyond USE would broaden its applicability.

%     \item \textbf{Real-time extensions:} Incorporating quantitative time constraints 
%     would enable verification of real-time properties, extending the approach to 
%     time-critical systems.

%     \item \textbf{Counterexample visualization:} Enhancing the plugin to provide 
%     more intuitive visualizations of counterexamples when temporal properties are 
%     violated would improve usability.
% \end{itemize}

% In conclusion, our TOCL+ approach successfully bridges the gap between standard 
% UML/OCL modeling and temporal verification, enabling modelers to specify and verify 
% complex behavioral properties without specialized temporal verification tools. While 
% there remain opportunities for improvement and extension, the approach provides a 
% practical solution to an important problem in model verification.


%%% Abstract
% \textbf{Abstract:} In Model-Driven Engineering (MDE), models serve as central artifacts 
% for abstracting and designing software systems. Modern software systems often need 
% to express and verify behaviors that involve temporal constraints and event-driven 
% conditions. The Unified Modeling Language (UML) and the Object Constraint Language
% (OCL) are widely used in MDE to model systems and specify constraints. While OCL is
% effective for defining structural and simple behavioral properties, it lacks the
% ability to express temporal constraints and event-based behaviors. This limitation
% makes it challenging to specify and verify dynamic aspects of systems. This thesis
% proposes an extension of OCL with temporal and event-based constructs to enhance
% its ability to express and verify behavioral properties. We implement this extension
% as a plugin, called TemporalOCL, for the UML-based Specification Environment (USE) tool.

% \textbf{Keywords}: \textit{Model-Driven Engineering, Object Constraints Language, Temporal Properties, Model Checking}

% Khoá luận của tôi sẽ được sử dụng một công cụ tên USE (UML-based Specification Environment), công cụ này hỗ trợ đặc tả mô hình theo cú pháp của uml và ocl. Công cụ còn có một plugin là Model Validator. USE Model Validator sẽ hỗ trợ validate mô hình trên các ocl  constraints đã định nghĩa, nó sẽ trả về một object diagram nếu mô hình có thể thoả mãn (SATISFIABLE case) hoặc UNSATISFIABLE. Nhưng mô hình uml và ocl chỉ đặc tả các thuộc tính cấu trúc mà không thể hiện được thuộc tính thời gian (system execution path). Tôi sẽ sử dụng một plugin khác của USE để chuyển đổi mô hình ban đầu thành dạng filmstrip (là mô hình gốc ban đầu và thêm các class Snapshot, OperationCall) để thể hiện system execution path thành dạng sequence of snapshot.
  % tôi đang viết khoá luận của mình với đề tài: "A SUPPORT TOOL TO SPECIFY AND VERIFY TEMPORAL PROPERTIES IN OCL". Mục tiêu chính của khoá luận là mở rộng ngôn ngữ ocl để hỗ trợ đặc tả thuộc tính thời gian (temporal properties) và sự kiện (event) từ đó kiểm tra tính đúng đắn của mô hình trong kỹ thuật model-driven engineering. Khoá luận của tôi sẽ được sử dụng một công cụ tên USE (UML-based Specification Environment), công cụ này hỗ trợ đặc tả mô hình theo cú pháp của uml và ocl. Công cụ còn có một plugin là Model Validator. USE Model Validator sẽ hỗ trợ validate mô hình trên các ocl  constraints đã định nghĩa, nó sẽ trả về một object diagram nếu mô hình có thể thoả mãn (SATISFIABLE case) hoặc UNSATISFIABLE. Nhưng mô hình uml và ocl chỉ đặc tả các thuộc tính cấu trúc mà không thể hiện được thuộc tính thời gian (system execution path). Tôi sẽ sử dụng một plugin khác của USE để chuyển đổi mô hình ban đầu thành dạng filmstrip (là mô hình gốc ban đầu và thêm các class Snapshot, OperationCall) để thể hiện system execution path thành dạng sequence of snapshot.. tôi muốn bạn đóng vai một giáo sư giúp tôi cấu trúc khoá luận của tôi. Hiện tại tôi đang dự kiến khoá luận sẽ có 5 phần chính. Phần 1 introduction sẽ giới thiệu về bài toán xoay quanh model-driven engineering và việc đảm bảo tính đúng đắn của mô hình và việc validate chúng. Nhưng hiện tại OCL có những giới hạn về việc đặc tả các thuộc tính thời gian và event. Phần 2: Foundational Knowledge sẽ tập trung và các kiến thức nền tảng (kiến thức về mde, ocl, USE tool, Filmstrip model, antlr4, TOCL operators, events trong OOP paradigm, ...). Phần 3: OCL Extension sẽ là phần trình bày về phương pháp mà tôi đã sử dụng (phần đóng góp chính). Khoá luận của tôi sử dụng TOCL operators (always P, sometime P, sometime P before Q, next P, previous P, ...) là kết quả của một bài báo khác, họ áp dụng cho mô hình STM (Snpashot Transition Model) (Khác với Filmstrip của tôi một chút). Tôi có tận dụng kết quả đó và điều chỉnh cho mô hình Filmstrip của tôi, bên cạnh đó  tôi còn mở rộng định nghĩa event cho ocl (isCalled(operation) =  operation (call/start/end) events in OOP paradigm, becomesTrue(P) = state change events in OOP paradigm). Phần 4: Implementation and Experiments, tôi thực hiện phương pháp đưa ra bằng việc phát triển một plugin của công cụ USE. Plugin này sẽ thực hiện việc chuyển đổi TOCL (định nghĩa dựa trên UML&OCL Application model) sang OCL thông thường (Định nghĩa trên Filmstrip model). Phần 5 là kết luận.  

%%% Version 3
% In Model-Driven Engineering (MDE), models serve as central artifacts 
% for abstracting and designing software systems. The Unified Modeling Language (UML) and 
% the Object Constraint Language (OCL) are widely used to define structural and behavioral 
% aspects of these models. As system models grow in size and complexity, the need for effective 
% validation and verification becomes critical. However, current UML modeling and verification tools 
% primarily focus on structural analysis and offer limited support for specifying dynamic system 
% behavior over time, due to OCL's foundation in first-order logic which cannot express properties across different model states. 
% This thesis proposes an extension of OCL with temporal and event-based constructs to enhance its 
% ability to express and verify behavioral properties. We implement this extension as a plugin, called 
% TemporalOCL, for the UML-based Specification Environment (USE). USE is a tool that supports the 
% development of UML models enhanced with OCL constraints and includes a model validator component for 
% analyzing structural properties of UML models. To enable practical validation of temporal specifications, 
% we employ a technique known as filmstripping, which transforms dynamic model properties into structural 
% constraints that can be analyzed using our USE model validator.

%%% Version 2
% In Model-Driven Engineering (MDE), models serve as central artifacts 
% for abstracting and designing software systems. The Unified Modeling Language (UML) and 
% the Object Constraint Language (OCL) are widely used to define structural and behavioral 
% aspects of these models. As system models grow in size and complexity, the need for effective 
% validation and verification becomes critical. However, current UML modeling and verification 
% tools primarily focus on structural analysis and offer limited support for specifying temporal behaviors and system 
% events, due to OCL’s foundation in first-order logic. This thesis proposes an extension of 
% OCL with temporal and event-based constructs to improve its expressiveness. We implement this 
% extension as a plugin, called TemporalOCL, for the UML-based Specification Environment (USE). 
% To enable validation of temporal properties, we adopt a technique known as filmstripping, 
% which allows dynamic aspects of models to be analyzed using the USE model validator.
  
%%% Version 1
% In Model-Driven Engineering (MDE), models are used as an abstraction
% of a system. The Unified Modeling Language (UML) and the Object Constraint Language (OCL) 
% is widely used for modeling and specifying constraints in MDE. As the size and complexity
% of models increase, the need for validation and verification becomes crucial. There are 
% a variety of verification and validation techniques available which mainly concentrate on structural
% characteristics of the model. Another problem is that OCL lacks the expressiveness to specify
% temporal properties and system events. In this thesis, we propose an extension of OCL
% to support the specification of temporal properties and events. We also present a plugin for the
% UML-based Specification Environment (USE) that allows users to specify and verify temporal properties. 


%%% INTRODUCTION
% The thesis is structured as follows:
% \begin{itemize}
%   \item \textbf{Chapter 1}: This chapter lays the foundation for the background of this thesis.
%   We explore theoretical concepts and tools that are used in this thesis.
  
%   \item \textbf{Chapter 2}: This chapter presents our OCL extension to specify temporal properties and events.

%   \item \textbf{Chapter 3}: This chapter describes the implementation and evaluation of the USE-TemporalOCL plugin. 

%   \item \textbf{Conclusion}: This chapter summarizes the contributions of this thesis and discusses future work.
% \end{itemize}

%%% Version 2
% Model-Driven Engineering (MDE) \cite{MDE} approaches system development 
% by emphasizing models over source code. Through modeling languages such as UML 
% and OCL, developers construct models that abstract the system complexity and serve as primary development artifacts. 
% These models can then be transformed into executable code, documentation, and test cases using automated 
% model transformation techniques. Techniques like validation and verification help 
% ensure the models’ quality, with structural aspects-represented by class and object 
% diagrams-being particularly important.

% The Unified Modeling Language (UML) offers a standard way to create diagrams that show 
% a system’s structure and behavior, while the Object Constraint Language (OCL) adds precise 
% rules to these models. OCL uses simple logic to define conditions that must always be true 
% or requirements for operations. However, it lacks the expressive power to model temporal properties 
% that involve different states of the model or depend on event occurrences. 
% This limitation stems from the absence of built-in constructs for handling time and events in OCL.

% In this thesis, we extend OCL to support the specification of temporal properties and 
% event-based behaviors, aiming to enhance its expressiveness in modeling dynamic system 
% aspects. 
% Our approach is implemented within the UML-based Specification Environment (USE) \cite{USE}, 
% a tool that supports specification, and validation of software systems using UML and OCL. 
% USE allows modelers to work with various UML diagrams and define constraints through invariants and 
% pre-/postconditions. It also provides validation and verification capabilities via the model 
% validator component, which translates models and properties into relational logic using Kodkod. 
% As a realization of this thesis, we worked on a plugin for USE that called USE-TemporalOCL, 
% which allows users to specify temporal properties and events in OCL.


%%% Version 1
% Model-driven engineering (MDE) takes a view on system development focusing on
% models rather than on code. Using modeling languages like UML and OCL, developers create models 
% that abstract the system’s complexity and serve as the central artifacts 
% in MDE. From these models, code, documentation, and test cases can be 
% automatically generated through model transformations. Techniques like validation 
% and verification help ensure the models’ quality, with structural aspects - 
% represented by class and object diagrams - being particularly important.

% Traditional software engineering, reliant on manual coding, 
% struggles with increasingly large, complex projects despite aiming for reliable and efficient systems. 
% This causes inefficiencies not from software requirements but from development 
% process limitations, driving the need for simpler, more flexible approaches. 
% To address these challenges, 
% Model Driven Engineering (MDE) \cite{MDE} offers a promising methodology 
% aimed at reducing accidental complexities by focusing on models over code. 
% Using modeling languages like UML and OCL, developers create models 
% that abstract the system’s complexity and serve as the central artifacts 
% in MDE. From these models, code, documentation, and test cases can be 
% automatically generated through model transformations. Techniques like validation 
% and verification help ensure the models’ quality, with structural aspects - 
% represented by class and object diagrams - being particularly important.

% Software engineering aims to build reliable and 
% efficient systems, but as projects become larger and more complex, 
% traditional development methods face many challenges. These methods 
% often rely heavily on manual coding, which can’t always keep up with 
% growing demands. As a result, problems and inefficiencies can occur—not 
% because of the software’s requirements, but because of how the 
% development process is handled. This shows the need for new approaches 
% that can make development easier and more flexible.
